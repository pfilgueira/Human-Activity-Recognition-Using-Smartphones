robust regression
 anova
 interaction plots

Logistic regression
 SA heart data
 
this data set is available in the elemstatlearn package
 fit a logistic regression model where the chd us the binary outcome, and all other variables are used as predictor variables
 
use the AIC criterion 
http://aimotion.blogspot.ie/2011/11/machine-learning-with-python-logistic.html?m=1
 
http://www.datarobot.com/blog/classification-with-scikit-learn/


------------------------
SHINY

Minimimal examples CB1 CB2
 Exercises
 Reactivity
 Conditional panelling
 HTML examples
 Mathjax and HTML


SAS 5 videos
 > Shiny Web Apps taxonomy map
 > Control Charts u charts
 > Operations Research Videos
 > Dublin R population dynamics

>
 > Data Analytics masters
 >
 > R Coursera ESL
 > Julia videoes london user group
 > sas videos
 > python probability talk
 >
 > hadoop rhadoop
 > predictive analytics
 >
 > pandas wes mckinney


Data Analytics masters
 
R 
Julia
 sas
 python
 
hadoop
 predictive anslytics
 
pandas
 practical data science


Attribute charts
 
Variable charts
 
multivariate charts
 
np charts
>
 > Part D checklist
 >
 > use of colours in plot

 > scatterplot abline
 residual plots
 cooks distance

Part D checklist
 
use of colours in plot
 scatterplot
 title and subtitle in plot
 Grubbs Test For Outlier
 log transformation
Set theory
 Set operations
 complement
 union intersection
 set difference
 venn diagrams
 8 Disjoint Regions
 membership tables
 
proof by truth tables
 AND
 OR
 NOT
 Set difference
 symmetric difference


We will use a simplistic system for interpreting significance values (i.e. p-values).
 
If a p value is less than 0.02 we reject the nyll hypothesis.
 If the p value is greater than 0.05 we fail to reject the null hypothesis
 If the pvalue us between the two thresholds then we deem the procedure to be inconclusive. 

The null hypothesis is that the true correlation coefficient is zero (which is to say, no linear relationship exists). 


Normal Distribution 
normal probability plot 

( qqplot boxplots histograms density plots)
 %-----------------------------------------------%
 skew and kurtosis describe function in the psych package 
grubbs test for outliers (outliers package)
 shapiro wilk test
 
anderson darling test (package normtest)
 
kolmogorov smirnov test for a single sample
 
correlation. Cor.test
 log transformation
 %----------------------------------------------%
 The t-tests 
variance test boxplots
 two sample tests 
paired t tests
 %-----------------------------------------------------%
 non parametric distributions
 ks-test
 wilcixon test for medians
Install.packages("packagename")
 library(packagename)
 #########
 data(datasetname) 
help(datasetname)
 
summary(datasetname)
 names(datasetname)
 
##########
 summary(datasetname$variablename)
 
## or 
attach(datasetname)
 summary(variablename)
 detach(datasetname)
 
##########
 
class(objectname)
 names(objectname)
 str(objectname)
 mode(objectname)
 
##########
 plot(variablename)
 plot(variablename1,variablename2)
 hist(variablename)


<hr>
Coursera data science specialization
 quality mgmt
 stata logistic regression
 susquehanna qr intern
 math biostats
 Dublin R







to Script 











ANOVA with R
 
The null hypothesis is that population mean is the same for each of the groups.
 
The alternative hypothesis is that there is at least  one group with a different population mean diffetent to the rest
 
the first tjing we do is to re-write the data in long form
 
Important :remember to construct tbe grouping variable as a factor. 


<hr>
Categorical variables are referred to by R (and in fact several branches of statistics as "factors". 

The categories for a factor are known as levels. 

Sex is an example of a factor variables, where the levels are Male and Female. 

Factors differ from character vectors. R woyld nit recognize a character variable as indicating membership of a category, rather it would just identify it as a piece of text..


 
-------------------------------------------

 
When computing confidence intervals or performing hypothesis tests, the most computationally complex component is the calculation of the standard error.
 
For each procedure, there is a standard error formula. These are usually given in an appendix to examination papers. It is strongly recommended that you familiarise yourself with these before your exam.



Scatterplots are useful for determining outliers. There is no single definition of an outlier, rather it depends on the type of analysis being used.


Medical statistic hypothesis testing
 
Type I error This describes a hypothesis test in which the null hypothesis was rejected, when it should not have been. 

Type II error This describes a hypothesis test in which the null hypothesis was rejected when it was in fact true.
 
-----------------------------------
 \begin{description}
 \item Specificity
 
\item Sensitivity
 \end{description}


Scatterplots are used to visualize the relationship between two continuous variables. The position of each point on the plot indicates the meaurement by both variables. Scatterplots are useful for determining the possible relationship between the variables.

Be mindful that the corration may be spurious. That is to say : The scatterplot indicates a relationship that does not really exist. Rather the points align in a linear manner on the scatterplot by chance. Special consideration should go into designing a scientific experiment to avoid such problems, including that there is a suitably large sample size. As the sample size gets less likely spurious correlation is less likely. 
