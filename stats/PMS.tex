\section{Probability and Mathematical Statistics}
\subsection*{Syllabus}

\begin{enumerate}	
\item	Exploratory Data Analysis
\item	Probability
\item	Random Variables and Probability Distributions
\item	Probability Generating Moments
\item	Discrete and Continuous Distributions
\item	Biavariate Data
\item	Central Limit Theorem
\item	
\item	
\item	
\item	
\item	ANOVA
\item	Conditional Expectations and Compound Distributions
\end{enumerate}	



The table below gives the number of thunderstorms reported in a particular summer
month by 100 meteorological stations.

Number of thunderstorms: &0& 1 &2 &3 &4 &5 \\ \hline
Number of stations: & 22 & 37 & 20 & 13 & 6 & 2 \\ \hline

\begin{itemize}
\item[(a)] Calculate the sample mean number of thunderstorms.
\item[(b)] Calculate the sample median number of thunderstorms.
\item[(c)] Comment briefly on the comparison of the mean and the median.
\end{itemize}
%------------------------------------------------%
% 12 Normal Approximation of Binomial
% PMS April 2006 
In a certain large population 45\% of people have blood group A. 
A random sample of 300 individuals is chosen from this population.
Calculate an approximate value for the probability that more than 115 of the sample
have blood group A.

If X is the number in the sample with group A, then X has a binomial (300, 0.45)
distribution, so
\[ E[X] = 300 \times 0.45 = 135 \] and 
\[ Var[X] = 300 \times 0.45 \times 0.55 = 74.25 \].
Then, using the continuity correction,

\[ P(X > 115) = P(X > 115.5)\]
\[ 1- \frac{115.5 - 135}{\sqrt{74.25} } \]
%-----------%
September 2006 Question 6
It is assumed that claims arising on an industrial policy can be modelled as a Poisson
process at a rate of 0.5 per year.
(i) Determine the probability that no claims arise in a single year. [1]
(ii) Determine the probability that, in three consecutive years, there is one or more
claims in one of the years and no claims in each of the other two years. [2]
(iii) Suppose a claim has just occurred. Determine the probability that more than
two years will elapse before the next claim occurs. [2]
[Total 5]
\[  P(X > 115.5) = 1 -\theta(-2.26) = \theta(2.26) = 0.99\]

%=========================================================% 
% PMS Autumn 2008 Question 1
The mean of a sample of 30 claims is \$5,200.
Six have mean of \$ 8000 (i.e. group 1)
Ten have mean of \$ 3100 (i.e. group 2)

Compute the mean for the remaining claims

\[\mbox{Total Costs} = (Cost for Group 1) + (Cost for Group 2) + (Cost from Group 3)\]

\begin{itemize}
\item Total Cost for all three groups : $\$5200 \times 30 = \$156000$
\item Cost for Group 1 : $\$8000 \times 6 = \$48000$
\item Cost for Group 2 : $\$3100 \times 10 = \$31000$
\end{itemize}

Necessarily the cost for group 3 is $\$77000$

The mean claim for group 3 is therefore

\[ \frac{\$77000}{14} = \$5500 \]

%=========================================================% 

% PMS SPring 2006 Question 6
Poisson/Binomial/Exponential

\begin{itemize}
\item  Poisson

Find P(X=0) for Poisson Mean (m=0.5)


\[ P(X=0) = \frac{e^{-0.5}}{0!}  = 0.606 \]


%=================%
\item Binomial

\[ { 3 \choose 1} \times (0.606)^2 times (1-0.606)^1 \]

= 0.434

%=================%
\item Exponential

No Claim in the next two years
= (0.606)^2 = 0.368

Time Until Next Claim

$\muT = 0.5$

$T \sim (exp(0.5)$

$P(XT >2) = exp(-1) = 0.368$

\end{itemize}
%============================================%
% PMS Spring 2008 Question 4 - Probability

Consider Two events A, B such that

\begin{itemize}
\item P(A) = 0.3
\item P(A \cap B) = 0.1
\end{itemize}

Find the minimum possible value of $P(A|B)$

\[ P(A|B) = \frac{P(A \cup B) }{P(B)}\]

\begin{itemize}
\item P(A \cup B) Maximum = 1
\item P(A \cup B) Minimum = 0.3
\end{itemize}

\subsection*{Minimum}

\[ P(A \cup B) = P(A) + P(B) - P(A \cap B)\]
\[ 0.3 = 0.3 + P(B) - 0.1 \]
\[ P(B) = 0.1\] 

\subsection*{Maximum}

\[ P(A \cup B) = P(A) + P(B) - P(A \cap B)\]
\[ 1 = 0.3 + P(B) - 0.1 \]
\[P(B) = 0.8\]

\[ P(A|B)_{MIN} = \frac{P(A \cup B) }{P(B)} = \frac{0.1}{0.8} = 0.125 \]
\[ P(A|B)_{MAX} = \frac{P(A \cup B) }{P(B)} = \frac{0.1}{0.1} = 1 \]
%============================================%
% PMS SPring 2006 Question 5
\section{Poisson Approximation}

n  = 25
p = 0.1, 0.2

Poisson approximation of Binomial ( letting $m=np$

\begin{itemize}
\item $m_1 = 2.5$
\item $m_2 = 5$
\end{itemize}

Find P(X\geq 5) 

$ P(X\leq 4)  = 1 - P(X\geq 5) 

From Tables 
0.89118
0.44049

(Rest : Compare to Real Answers)

%========================================================%
% PMS Spring 2006 Question 3

A random sample of 10 is taken from a normak distribution of $\mu=20$ and $\sigma^2=1$. Let $s^2$ be the sample variance.

Find $P(S^2>1)$

\subsection*{Solution}

\[ \frac{(n-1)S^2}{\sigma^2} \sim \chi^2_{n-1}\]

\[ 9S^2 \sim \chi^2_{9}\]


\[ P(S^2>1) = P(\chi^2_{9}>9) = 1.05627 = 0.437\]

%========================================================%

PMS Spring 2008 
\[f(x,y) = \frac{4}{3}(1-xy) \mbox{   }0<x<1,0<y<1  \]

The Marginal PDF of X and Y is given by 

\[f(x) = \frac{2}{3}(2-x) \mbox{   }0<x<1 \]
\[f(x,y) = \frac{2}{3}(2-y) \mbox{   }0<y<1  \]

Show that the conditional expectation of $Y$ given $X$ is given by

\[ f(y|x) = \frac{2(1-xy}{2-7} \mbox{   }0<y<1\]

\textbf{Solution}

\[ f(y|x) = \frac{f(x,y)}{f(x)} = \frac{\frac{4}{3}(1-xy)}{\frac{2}{3}(2-x)} = \frac{2(1-xy)}{2-x}\]

%========================================================%

% PMS Autumn 2009 Question 8

Find $E[X|Y=2]$

% TABLE HERE
    & X=0  & X=1  & X=2  &              \\ \hline
Y=1 & 0.15 & 0.2  & 0.25 & P(Y=1) = 0.6 \\ \hline
Y=2 & 0.05 & 0.15 & 0.20 & P(Y=2) = 0.4 \\ \hline
    & P(X=0) = 0.2  & P(X=1) = 0.35  & P(X=2)=0.45  &              \\ \hline
    
    
\textbf{Solution}
\[   \frac{(0 \times 0.05) + (1 \times 0.15)+(2 \times 0.2) }{0.4}  = \frac{0.55}{0.4}  \] 

$E[X|Y=2] = 1.375$
