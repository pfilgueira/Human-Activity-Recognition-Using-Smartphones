\documentclass[12pt]{article}

% http://www.cims.nyu.edu/~regev/teaching/discrete_math_fall_2005/dmbook.pdf

% http://math.about.com/gi/o.htm?zi=1/XJ&zTi=1&sdn=math&cdn=education&tm=17&f=10&tt=14&bt=8&bts=8&zu=http%3A//www.ping.be/~ping1339/tel.htm


%opening
\title{Discrete Mathematics}
\author{Kevin O'Brien}

\begin{document}

\maketitle

\begin{abstract}
Discrete Mathematics Udemy Course
\end{abstract}


%-------------------------------------------------%
\section{Set Theory}

\begin{itemize}
\item Set Operations
\item Power Sets
\item Subsets
\item Venn Diagrams
\end{itemize}

\subsection*{Introduction to Sets}
%http://people.umass.edu/partee/NZ_2006/Set%20Theory%20Basics.pdf
A set is a collection of objects which are called the members or elements of 
that set. If we have a set we say that some objects belong (or do not belong) to this set, are
(or are not) in the set. We say also that sets consist of their elements. 


\subsection*{Specifying Sets}

There are three main ways to specify a set: 

\begin{itemize}
\item[(1)] by listing all its members (list notation); 
\item[(2)] by stating a property of its elements (predicate notation); 
\item[(3)] by defining a set of rules which generates (defines) its members (recursive rules).
\end{itemize}

\begin{itemize}
\item $\emptyset$ The empty set
\item $\mathcal{p}(S)$ The power set
\end{itemize}


%-------------------------------------------------%
\section{Counting}
\begin{itemize}
\item Combinations
\item Permutations
\item The choose operator
\end{itemize}

\subsection*{Factorial Function}
The product of the positive integers from 1 to n inclusive is denoted by $n!$, read “n factorial.” Namely:
\[n! = 1 \times 2 \times 3 \times \ldots \times (n−2) \times(n−1)\times n\] 

Accordingly, 1! = 1 and $n! = n(n − l)!$. 

It is also convenient to define 0! = 1.


\[ {n \choose k} = \frac{n!}{k! \times (n-k)!} \]


\subsection*{Permutation}
Any arrangement of a set of n objects in a given order is called a permutation of the object (taken all at a time).

\newpage
%-------------------------------------------------%
\section{Logic}
\begin{itemize}
\item Important Logical Operators
\item Conditional Connectives
\item Logic Tables
\item Using Logic Tables for Proofs
\item Logical Gate Networks
\end{itemize}



\subsection*{Important Logical Operators}

\begin{description}
\item[$p \vee q$] $p$ and $q$
\item[$p \wedge q$] $p$ or $q$
\item[$\neg p$] Not P
\end{description}


\subsection*{Logic Gates}

\begin{description}
\item AND gates
\item OR gates
\item XOR gates
\item NOT gates
\end{description}
\newpage
%-------------------------------------------------%
\section{Numbers and Number Systems}
\begin{itemize}
\item Binary Numbers
\item Hexadecimal Numbers
\item Octal Numbers and Base 5 Numbers
\end{itemize}

\subsection*{Hexadecimal Numbers}
Hexadecimal numbers are commonly used to represent memory addresses in computer systems.
 
 
Hexadecimal Numbers uses sixteen distinct symbols, most often the symbols 0–9 to represent values zero to nine, and A, B, C, D, E, F (or alternatively a–f) to represent values ten to fifteen. For example, the hexadecimal number 2AF3 is equal, in decimal, to $(2 \times 16^3)$ + $(10 \times 16^2)$ + $(15 \times 16^1$) + ($3 \times 16^0$), or 10995.

%-------------------------------------------------%
\newpage
\section{Functions (1)}
\begin{itemize}
\item Logarithms
\item Exponentials
\item The Absolute Value Functions
\item Trigonometric Functions
\item The floor and Ceiling functions
\end{itemize}
%-------------------------------------------------%
\section{Functions (2)}
\begin{itemize}
\item Arrow Diagrams
\item One-to-One functions
\item Onto Functions
\end{itemize}

%------------------------------------------------
\section{Graph Theory}
\begin{itemize}
\item Introduction to Graph Theory
\item KEy Terms and Definitions in Graph Theury
\item Isomorphism
\item Digraphs
\end{itemize}


\section{Proof by Induction}


\section{Relations and Functions}
\begin{itemize}
\item Ordered Pairs
\item Cartesian Products
\end{itemize}

\subsection*{Partial Ordering Relations}
A relation R on a set S is called a partial ordering or a partial order of S if R is reflexive, antisymmetric, and
transitive. A set S together with a partial ordering R is called a partially ordered set or poset.

\end{document}