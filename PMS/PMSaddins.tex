\documentclass[]{article}
\usepackage{amsmath}


\begin{document}



\begin{itemize}
\item The gamma distribution can be parameterized in terms of a shape parameter $\alpha$ = k and an inverse scale parameter $\beta$, called a rate parameter.
\item A random variable X that is gamma-distributed with shape $\alpha$ and rate $\beta$ is denoted
\[
X \sim  \textrm{Gamma}(\alpha,\beta)
\]
\end{itemize}

{
\LARGE
\[\scriptstyle\mathbf{E}[ X] = \frac{\alpha}{\beta}\]
}

{
\LARGE
\[
\scriptstyle \operatorname{Var}[ X] = \frac{\alpha}{\beta^2}
\]
}
\bigskip
\begin{itemize}
\item
By the central limit theorem, if $\alpha$ is large, then gamma distribution can be approximated by the normal distribution with mean and standard deviation\[\mu = \frac{\alpha}{\beta}\] 
 \[\sigma=\frac{\sqrt{\alpha}}{\beta}\]


\item That is, the distribution of the variable  
{
\LARGE
\[Z=\frac{X-\frac{\alpha}{\beta}}{\frac{\sqrt{\alpha}}{\beta}}\]} tends to the standard normal distribution as $ \alpha \longrightarrow \infty$.
\end{itemize}
\newpage
{
\Huge
\[
\scriptstyle \mbox{ Point Estimate } \pm  \left[ \operatorname{ Quantile } \times \operatorname{Standard Error } 
 \right] 
\]}
\end{document}
