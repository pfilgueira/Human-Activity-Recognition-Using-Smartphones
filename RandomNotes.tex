VaVaVoom

The plant in Austria produces 80% of the cars.

The plant in Belgium produces 20% of the cars.

A randomly chosen car was build at Austrian plant

A randomly chosen car was built at the Belgian plant

S: A randomly chosen car has standard





4 & 6 & 8 & 9 & 17 & 17 & 18 & 19 & 20 & 22 \\

22 & 27 & 28 & 29 & 31 & 35 & 38 & 39 & 40 & 46 \\

48 & 56 & 56 & 57 & 57 & 58 & 58 & 60 & 61 & 62 \\

64 & 66 & 68 & 69 & 74 & 75 & 78 & 79 & 80 & 82 \\


lower fence?
Upper fence?

Any values above or below fences?






----------------------------------------------------------------------

A motor dealership which specializes in agricultural machinery sells on vehicle every 2 days, on average


In this question the unit period is one day. The company expects to sell, on average, 0.5 vehicles every day.

The Poisson mean $m$ is therefore 0.5.


P(X \geq 1)


Go to your Poisson tables, and search for the $m=0.5$ column.

We are interested in the probability of \textbf{exactly} one vehicle sold on a particular day.

From the tables we can easily work out P(X \geq 1), but this is probability of one or more vehicles being sold.

This is not the same thing.

P(X \geq 1) = P(X =1) + P( X=2) + P(X=3) + \ldots

P(X \geq 1) = P(X=1) + P(X \geq 2)

From tables
P(X \geq 1)
P(X \geq 2)


Six day working week?
our unit period is now six days.
How many vehicles do we expect to sell in 6 days?
answer = 3

$m=3$


P(X\geq 4)


e^{-6/5} = 0.3011942

e^{-4/5} = 0.449329

e^{-5/5} = 0.3678794


P(B|A) = { P(A \mbox{ and } B) \over P(A) }
 
P(A|B) = { P( A \mbox{ and }B) \over P(B) }






The exponential distribution
The average lifespan ppf a laptop is5 year. You may assume that the lifespan of laptop computers follows an exponential distribution.

What is the probability that the lifespan of the laptop will be at least 6 years.

What is the probability that the lifespan of the laptop will not exceed 4 years.

What is the probability that the lifespan of the laptop will be between 5 years and 6 years.



It is a one tailed test

$H_o$  : $\mu = 80 $
$H_a$  : $\mu \neq 80$ 

The significance level is 5% (or 0.05)

what is the column to use?

what is the degrees of freedom 
Is it a large sample or a small sample?


\bar{x} = 

s^2 = 16
CV = 


\sqrt{3}{1.09 \times 1.08 \times 1.07}


\sqrt{ \frac{\hat{p} 1- \hat{p}}{n} }


\sqrt{ \frac{\hat{p_1} 1- \hat{p_1}}{n_1} + \frac{\hat{p_2} 1- \hat{p_2}}{n_2}}


H_o
H_a

The F-test
H_0: Both variances are equal
H_a : The variances are different.

Compute the test statistic.

Divide the larger variance by the smaller variance.

The degrees of freedom are as follows
 
\nu_1 size of sample with larger variance
\nu_2 size of sample with smaller variance


There are 5 values tabulated
We use the one for a significance level of 0.05
Carefully read the tables.


Binomial Distribution

There are n independent trials
The probability of a success is

\sum_x
\sum_y
\sum_xy
\sum_x^2
\sum_y^2#

n=10






Fading
>Transitions
>Fade Thru Black
>Drag and Drop
>Right Click on Arrow - Change transition from 3 to 5
>Both First Slide and End Slide
>Change the final slide to 2 second.
