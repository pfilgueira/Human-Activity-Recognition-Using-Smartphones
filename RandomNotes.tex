
%------------------------------------------------------------------------%
\frame{\frametitle{Problem}
\large
A  government agency carries out a large-scale random survey of public
attitudes towards the recession. 200 of the 900 workers surveyed indicated they
were worried about losing their job. 



}



\begin{frame}
\frametitle{Introduction}
The objective of the survey is to obtain an assessment of the views or opinions of students studying in the Faculty of Business and Accounting studies at a specific university.
\vspace{0.4cm}
We would like to know p which is the ``proportion of successes". For instance, p could be:
\begin{itemize} \item the proportion of U.S. citizens that support Obama,
\item  the proportion of smokers among adults age 18 or over,
\item the proportion of people worldwide infected by the H1N1 virus.
\end{itemize}
\end{frame}

%--------------------------------------------------------------------------------------------------------------------------------------------------------%
\frame{
\frametitle{The Sample Proportion}
\begin{itemize}
\item The sample proportion is computed from the sample.
\item It is the proportion of `successes' in the sample.
\item Suppose we are interested in the number of people who like geese.
\end{itemize}
}


%------------------------------------------------------------------------%
\frame{

Using these values, we can calculate the standard error with this expression.

\vspace{0.1cm}
\[
S.E. (\hat{P}) = \sqrt{ { 0.60 \times 0.40 \over 400}}
\]

\vspace{0.1cm}

However, it is easier to perform such calculates when working in terms of percentages.

\vspace{0.1cm}
\[
S.E. (\hat{P}) = \sqrt{ { 60 \times 40 \over 400}}  \;[\%]
\]

}

%------------------------------------------------------------------------%
\end{document}


%------------------------------------------------------------------------%
\frame{\frametitle{Problem}
A pet food supplier is studying the difference between two of its stores. It is
particularly interested in the time it takes before customers receive the products
they have ordered. Using standard notation, the data of delivery times from
the two stores is as follows:
\begin{tabular}{|c|c|c|}
  \hline
   & Store A  & Store A  \\
  $\bar{x}$ & 34.3 days & 38.6 days \\
  S & 2.5 days & 3.4 days \\
  Sample Size & 41 & 31 \\
  \hline
\end{tabular}

\begin{itemize}
\item Use an appropriate hypothesis test to see if there is a difference in the
average delivery times for the two stores. \item Test at two appropriate levels
and comment on your findings. \end{itemize}
}



%------------------------------------------------------------------------%
\frame{

\begin{itemize}
\item $\mbox{Test Statistic}= { 150\; - \;0 \over 30 }$\\
\item $\mbox{Statistic} : \hat{P}_{1} - \hat{P}_{2}$\\
\end{itemize}

}

%------------------------------------------------------------------------%
\frame{\frametitle{Problem}


\[
P (X > 15) = e ^{-15\lambda}
= e ^{-3 / 2}
= 0.22
\]


What is the probability that a customer will spend more than 15 minutes in the bank given that he is still in the bank after 10 minutes?
\[
P (X > 15|X > 10) = P (X > 5) = e
= e ^{-3 / 2}
= 0.604
\]


}







}
%------------------------------------------------------------------------%
\frame{\frametitle{Problem}

Suppose we want to estimate the average weight of an adult american male. We draw a random sample of 100 men from the population  and weigh them.\\ \vspace{0.3cm} We find that the average man in our sample weighs 180 pounds, and the standard deviation of the sample is 30 pounds.\\ \vspace{0.3cm}What is the 95\% confidence interval?



}
%------------------------------------------------------------------------%
\frame{\frametitle{Problem}

\begin{itemize}
\item
\textbf{Identify a sample statistic} - Since we are trying to estimate the mean weight in the population, we choose the mean weight in our sample (180) as the sample statistic.


\item \textbf{Select a confidence level}  -In this case, the confidence level is defined for us in the problem. We are working with a 95\% confidence level.


\item \textbf{Find the margin of error} - Previously, we described how to compute the margin of error.
\end{itemize}



}
%------------------------------------------------------------------------%
\frame{

Using these values, we can calculate the standard error with this expression.

\vspace{0.1cm}
\[
\mbox{Std. Error}(\bar{X})  = \sqrt{{30^2\over 100}} = \sqrt{9}
 = 3\]

\vspace{0.1cm}

The Standard Error is 3lbs.



}



%------------------------------------------------------------------------%

\begin{frame}\frametitle{Outline of the Survey}
The objective of the survey is to obtain an assessment of the views or opinions of students studying in the Faculty of Business and Accounting studies at a specific university.

\vspace{0.4cm}

The Survey is broken into three parts - A,B and C. \\ \vspace{0.2cm}

A - Questions in this section are of ``Likert'' type. The data obtained here is ordinal (Categorical) although we treat it as if it were interval (Numerical) for the analysis.\\
\vspace{0.2cm}
B - One question asking people to indicate what School they are from - nominal (Categorical) data.\\
\vspace{0.2cm}
C - Another Likert question.
\end{frame}




%------------------------------------------------------------------------%
\end{document}VaVaVoom

The plant in Austria produces 80% of the cars.

The plant in Belgium produces 20% of the cars.

A randomly chosen car was build at Austrian plant

A randomly chosen car was built at the Belgian plant

S: A randomly chosen car has standard





4 & 6 & 8 & 9 & 17 & 17 & 18 & 19 & 20 & 22 \\

22 & 27 & 28 & 29 & 31 & 35 & 38 & 39 & 40 & 46 \\

48 & 56 & 56 & 57 & 57 & 58 & 58 & 60 & 61 & 62 \\

64 & 66 & 68 & 69 & 74 & 75 & 78 & 79 & 80 & 82 \\


lower fence?
Upper fence?

Any values above or below fences?






----------------------------------------------------------------------

A motor dealership which specializes in agricultural machinery sells on vehicle every 2 days, on average


In this question the unit period is one day. The company expects to sell, on average, 0.5 vehicles every day.

The Poisson mean $m$ is therefore 0.5.


P(X \geq 1)


Go to your Poisson tables, and search for the $m=0.5$ column.

We are interested in the probability of \textbf{exactly} one vehicle sold on a particular day.

From the tables we can easily work out P(X \geq 1), but this is probability of one or more vehicles being sold.

This is not the same thing.

P(X \geq 1) = P(X =1) + P( X=2) + P(X=3) + \ldots

P(X \geq 1) = P(X=1) + P(X \geq 2)

From tables
P(X \geq 1)
P(X \geq 2)


Six day working week?
our unit period is now six days.
How many vehicles do we expect to sell in 6 days?
answer = 3

$m=3$


P(X\geq 4)


e^{-6/5} = 0.3011942

e^{-4/5} = 0.449329

e^{-5/5} = 0.3678794


P(B|A) = { P(A \mbox{ and } B) \over P(A) }
 
P(A|B) = { P( A \mbox{ and }B) \over P(B) }






The exponential distribution
The average lifespan ppf a laptop is5 year. You may assume that the lifespan of laptop computers follows an exponential distribution.

What is the probability that the lifespan of the laptop will be at least 6 years.

What is the probability that the lifespan of the laptop will not exceed 4 years.

What is the probability that the lifespan of the laptop will be between 5 years and 6 years.



It is a one tailed test

$H_o$  : $\mu = 80 $
$H_a$  : $\mu \neq 80$ 

The significance level is 5% (or 0.05)

what is the column to use?

what is the degrees of freedom 
Is it a large sample or a small sample?


\bar{x} = 

s^2 = 16
CV = 


\sqrt{3}{1.09 \times 1.08 \times 1.07}


\sqrt{ \frac{\hat{p} 1- \hat{p}}{n} }


\sqrt{ \frac{\hat{p_1} 1- \hat{p_1}}{n_1} + \frac{\hat{p_2} 1- \hat{p_2}}{n_2}}


H_o
H_a

The F-test
H_0: Both variances are equal
H_a : The variances are different.

Compute the test statistic.

Divide the larger variance by the smaller variance.

The degrees of freedom are as follows
 
\nu_1 size of sample with larger variance
\nu_2 size of sample with smaller variance


There are 5 values tabulated
We use the one for a significance level of 0.05
Carefully read the tables.


Binomial Distribution

There are n independent trials
The probability of a success is

\sum_x
\sum_y
\sum_xy
\sum_x^2
\sum_y^2#

n=10






Fading
>Transitions
>Fade Thru Black
>Drag and Drop
>Right Click on Arrow - Change transition from 3 to 5
>Both First Slide and End Slide
>Change the final slide to 2 second.
