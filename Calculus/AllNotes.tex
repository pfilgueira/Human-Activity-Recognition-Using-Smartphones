\documentclass[12pt, a4paper]{article}
\usepackage{natbib}
\usepackage{vmargin}
\usepackage{graphicx}
\usepackage{epsfig}
\usepackage{subfigure}
%\usepackage{amscd}
\usepackage{amssymb}
\usepackage{amsbsy}
\usepackage{amsthm, amsmath}
%\usepackage[dvips]{graphicx}

\renewcommand{\baselinestretch}{1.8}

% left top textwidth textheight headheight % headsep footheight footskip
\setmargins{3.0cm}{2.5cm}{15.5 cm}{23.5cm}{0.5cm}{0cm}{1cm}{1cm}

\pagenumbering{arabic}


\begin{document}
\author{Kevin O'Brien}
\tableofcontents

\subsection{Example}
Find $h(t)$ when $h(t) = f*g(t)$, with $f(t)= t$ and $g(t)=
t^2$.\newline

\begin{eqnarray}
f(t) = t \quad \Leftrightarrow \quad F(S)= \frac{1}{S^2}
 \nonumber\\
g(t) = t^2 \quad \Leftrightarrow \quad G(S)= \frac{2}{S^3}
 \nonumber\\
H(S) = F(S)\times G(S) = \frac{2}{S^5}
 \nonumber\\
(H(S) \mbox{ is in form }  k\frac{n!}{S^{n+1}} )
 \nonumber
\end{eqnarray}

With $n=4$, $n!= 4! = 24$. Solving for $k$, $k \times n! = 2$.
Therefore $k=\frac{1}{12}$. The solution is $\cal{L}^{-1}[H(S)]$

\section{Period of a trigonomteric function}
Period of a function is denoted $2l$. (Sometimes it is denoted as
$L$, with $L=2l$). \newline When given a trigonometric function in
form $f(t) = Cos(kx)$ or $f(t)= Sin(kx)$, the period of the
function can be calculated as follows:

\begin{eqnarray}
2l = \frac{2 \pi}{k}\nonumber
\end{eqnarray}


\subsection{Example}
\begin{eqnarray}f(t) = Cos(\frac{2 \pi x}{3}) \nonumber\\
2l \quad=\frac{2\pi}{(\frac{2\pi}{3})}\quad =
\frac{1}{(\frac{1}{3})}\quad= \textbf{3}\nonumber
\end{eqnarray}

\subsection{Example}
\begin{eqnarray}f(t) = Sin(\frac{5x}{2}) \nonumber\\
2l \quad=\frac{2\pi}{(\frac{5}{2})}\quad = \frac{4\pi}{5}\nonumber
\end{eqnarray}
\section{Curl}
In vector calculus, the curl is a vector operator that describes the infinitesimal rotation of a 3-dimensional vector field. At every point in the field, the curl of that field is represented by a vector. The attributes of this vector (length and direction) characterize the rotation at that point.
%--------------------------------------------------%
\section{Laplacian Operator}
\[\Delta f = \nabla^2 f = \nabla \cdot \nabla f\]

%--------------------------------------------------%
\section{Notation}
\[ \begin{vmatrix} \mathbf{i} & \mathbf{j} & \mathbf{k} \\
{\frac{\partial}{\partial x}} & {\frac{\partial}{\partial y}} & {\frac{\partial}{\partial z}} \\
 F_x & F_y & F_z \end{vmatrix}\]

\[\left(\frac{\partial F_z}{\partial y}  - \frac{\partial F_y}{\partial z}\right) \mathbf{i} + \left(\frac{\partial F_x}{\partial z} - \frac{\partial F_z}{\partial x}\right) \mathbf{j} + \left(\frac{\partial F_y}{\partial x} - \frac{\partial F_x}{\partial y}\right) \mathbf{k}\]

 %--------------------------------------------------%
\section{Example}
\[f(x,y,z)= \ 2x+3y^2-\sin(z)\]

\[ \nabla f=
\frac{\partial f}{\partial x} \mathbf{i} +
\frac{\partial f}{\partial y} \mathbf{j} +
\frac{\partial f}{\partial z} \mathbf{k}
 = 2\mathbf{i}+ 6y\mathbf{j} -\cos(z)\mathbf{k}\]

\section{Stoke's Theorem}
\[
\iint_{\Sigma} \nabla \times \mathbf{F} \cdot d\mathbf{\Sigma} = \oint_{\partial\Sigma} \mathbf{F} \cdot d \mathbf{r}, \]

\section{Laplace Transforms }
If $g(t)=k \times f(t)$ then $G(S) = k \times F(S)$ where $k$ is a
constant. $\mathcal{f(t)}=F(S)$.
\begin{eqnarray}
f(t) &=& (t+1)^2\\
&=& t^2 +2t +1 \nonumber
\end{eqnarray}

\section{Inverse Laplace Transforms 2}

The denominator has form $S^2 - 2aS + a^2 + k$ which is equivalent
to $(S-a)^2 + k$. Therefore $G(S)$ will have form $F(S-a)$
\newline
The function $G(S)$ may have the form $\frac{S+D}{S^2 +(C+D)S +
CD}$, where C and D are constants. This expression simplifies
$\frac{S+D}{(S+C)(S+D)}$ and again to $\frac{1}{S+C}$. The inverse
laplace transform $g(t)$ can be easily determined.
\section{Convolution}
We are asked to find a function h(t) which is the convolution of
two given functions $f(t)$ and $g(t)$. i.e $h(t)=h*g(t)$.\newline
Importantly $H(S) = F(S)\times G(S)$. We determine the laplace
transforms, F(S) and G(S), and multiply them to determine H(S). We
then find the inverse Laplace transform of H(S) to yield our
solution.
\subsection{Example}
Find $h(t)$ when $h(t) = f*g(t)$, with $f(t)= e^{t}$ and $g(t)=
e^{-t}$.\newline

\begin{eqnarray}
f(t) = e^{t} \quad \Leftrightarrow \quad F(S)= \frac{1}{S-1}
 \nonumber\\
g(t) = e^{-t} \quad \Leftrightarrow \quad G(S)= \frac{1}{S+1}
 \nonumber\\
H(S) = F(S)\times G(S) = \frac{1}{(S-1)(S+1)}
 \nonumber
\end{eqnarray}

\subsection*{Partial Differentiation}
 \[\frac{\partial f}{\partial x}\]

\[z = f(x, y) = \,\! x^2 + xy + y^2.\]
\[\frac{\partial z}{\partial x} = 2x+y\]
\[\frac{\partial z}{\partial x} = 3\]
\section{Laplace Transforms Using 1st Shifting Theorem}
\begin{eqnarray}
 g(t) = e^{at}f(t) \quad \Leftrightarrow \quad G(S)= F(S-a) \nonumber
\end{eqnarray}
The function $g(t)$ is presented in a form whereby $a$ and $f(t)$
are easily identifiable. First determine $F(S)$ by finding the
Laplace transform of $f(t)$. Then replace all $S$ terms with
$S-a$.

\section{Laplace Transforms Using 2nd Shifting Theorem}

\begin{eqnarray}
 g(t) = u^{a}f(t-a) \quad \Leftrightarrow \quad G(S)= e^{-aS}F(S) \nonumber
\end{eqnarray}
The function $g(t)$ is presented in a form whereby $a$ and
$f(t-a)$ are easily identifiable. ($U_{a}(t)$ is called the unit
step function). First determine $f(t)$ by replace all $t-a$ terms
in $f(t-a)$ with $t$. Then calculate the laplace transform of
$f(t)$ i.e. $F(S)$. The solutions is in form $G(S)= e^{-aS}F(S)$.

\section{Inverse Laplace Transforms (2 questions) }
Partial fraction expansion is used in questions 4 and 5.


\section{Even and Odd Function}
Even Functions: $Cos(X)$ ,$|X|$ (i.e absolute value of $X$) and
$X^2$, $X^4$ etc
\newline
Odd Functions: $Sin(X)$, $X$, $X^3$ etc
\\
Functions that are products of two even functions are also
\textbf{even} functions.
\newline
Functions that are products of two odd functions are
\textbf{even} functions. (e.g $X \times X^3 = X^4$)
\newline
Functions that are products of an even function and an
odd function are \textbf{odd} functions.
\section{Fourier Series - determining the arguments}
Given a period $2l$, we must determine the form of the fourier
series. $sin( \frac{n x \pi}{l})$
\section{Fourier Series}
X

MA4005 Syllabus
\begin{itemize}
\item Functions of several variables and partial differentiation.
\item The indefinite integral. Integration techniques: of standard functions, by substitution, by parts and using partial fractions.
\item The definite integral. Finding areas, lengths, surface areas, volumes, and moments of inertial.
\item Numerical integration: trapezoidal rule, Simpson's rule.
\item Ordinary differential equations.
\item First order including linear and separable. Linear second order equations with constant coefficients.
\item Numerical solution by Runge-Kutta. The Laplace transform: tables and theorems and solution of linear ODEs.
\item Fourier series: functions of arbitary period, even and odd functions, half-range expansions.
\item Application of Fourier series to solving ODEs.
\item Matrix representation of and solution of systems of linear equations.
\item Matrix algebra: invertibility, determinants.
\item Vector spaces: linear independence, spanning, bases, row and column spaces, rank.
\item Inner products: norms, orthogonanality. Eigenvalues and eignenvectors.
\item Numerical solution of systems of linear equations. Gauss elimination, LU decomposition, Cholesky decomposition,
iterative methods. Extension to non-linear systems using Newton's method.
\end{itemize}
%-------------------------------%
\subsection{convolution}

\[ (f * g)(t) = \int_0^t f(\tau)g(t-\tau)\,d\tau= F(s) \cdot G(s) \]



\section{Laplace Transforms Using 1st Shifting Theorem}
\begin{eqnarray}
 g(t) = e^{at}f(t) \quad \Leftrightarrow \quad G(S)= F(S-a) \nonumber
\end{eqnarray}
The function $g(t)$ is presented in a form whereby $a$ and $f(t)$
are easily identifiable. First determine $F(S)$ by finding the
Laplace transform of $f(t)$. Then replace all $S$ terms with
$S-a$.

\section{Laplace Transforms Using 2nd Shifting Theorem}

\begin{eqnarray}
 g(t) = u^{a}f(t-a) \quad \Leftrightarrow \quad G(S)= e^{-aS}F(S) \nonumber
\end{eqnarray}
The function $g(t)$ is presented in a form whereby $a$ and
$f(t-a)$ are easily identifiable. ($U_{a}(t)$ is called the unit
step function). First determine $f(t)$ by replace all $t-a$ terms
in $f(t-a)$ with $t$. Then calculate the laplace transform of
$f(t)$ i.e. $F(S)$. The solutions is in form $G(S)= e^{-aS}F(S)$.

\section{Inverse Laplace Transforms (2 questions) }
Partial fraction expansion is used in questions 4 and 5.
\section{Inverse Laplace Transforms 2}

The denominator has form $S^2 - 2aS + a^2 + k$ which is equivalent
to $(S-a)^2 + k$. Therefore $G(S)$ will have form $F(S-a)$
\newline
The function $G(S)$ may have the form $\frac{S+D}{S^2 +(C+D)S +
CD}$, where C and D are constants. This expression simplifies
$\frac{S+D}{(S+C)(S+D)}$ and again to $\frac{1}{S+C}$. The inverse
laplace transform $g(t)$ can be easily determined.
\section{Convolution}
We are asked to find a function h(t) which is the convolution of
two given functions $f(t)$ and $g(t)$. i.e $h(t)=h*g(t)$.\newline
Importantly $H(S) = F(S)\times G(S)$. We determine the laplace
transforms, F(S) and G(S), and multiply them to determine H(S). We
then find the inverse Laplace transform of H(S) to yield our
solution.
\subsection{Example}
Find $h(t)$ when $h(t) = f*g(t)$, with $f(t)= e^{t}$ and $g(t)=
e^{-t}$.\newline

\begin{eqnarray}
f(t) = e^{t} \quad \Leftrightarrow \quad F(S)= \frac{1}{S-1}
 \nonumber\\
g(t) = e^{-t} \quad \Leftrightarrow \quad G(S)= \frac{1}{S+1}
 \nonumber\\
H(S) = F(S)\times G(S) = \frac{1}{(S-1)(S+1)}
 \nonumber
\end{eqnarray}
\subsection{Example}
Find $h(t)$ when $h(t) = f*g(t)$, with $f(t)= t$ and $g(t)=
t^2$.\newline

\begin{eqnarray}
f(t) = t \quad \Leftrightarrow \quad F(S)= \frac{1}{S^2}
 \nonumber\\
g(t) = t^2 \quad \Leftrightarrow \quad G(S)= \frac{2}{S^3}
 \nonumber\\
H(S) = F(S)\times G(S) = \frac{2}{S^5}
 \nonumber\\
(H(S) \mbox{ is in form }  k\frac{n!}{S^{n+1}} )
 \nonumber
\end{eqnarray}

With $n=4$, $n!= 4! = 24$. Solving for $k$, $k \times n! = 2$.
Therefore $k=\frac{1}{12}$. The solution is $\cal{L}^{-1}[H(S)]$

\section{Period of a trigonomteric function}
Period of a function is denoted $2l$. (Sometimes it is denoted as
$L$, with $L=2l$). \newline When given a trigonometric function in
form $f(t) = Cos(kx)$ or $f(t)= Sin(kx)$, the period of the
function can be calculated as follows:

\begin{eqnarray}
2l = \frac{2 \pi}{k}\nonumber
\end{eqnarray}


\subsection{Example}
\begin{eqnarray}f(t) = Cos(\frac{2 \pi x}{3}) \nonumber\\
2l \quad=\frac{2\pi}{(\frac{2\pi}{3})}\quad =
\frac{1}{(\frac{1}{3})}\quad= \textbf{3}\nonumber
\end{eqnarray}

\subsection{Example}
\begin{eqnarray}f(t) = Sin(\frac{5x}{2}) \nonumber\\
2l \quad=\frac{2\pi}{(\frac{5}{2})}\quad = \frac{4\pi}{5}\nonumber
\end{eqnarray}

\section{Even and Odd Function}
Even Functions: $Cos(X)$ ,$|X|$ (i.e absolute value of $X$) and
$X^2$, $X^4$ etc
\newline
Odd Functions: $Sin(X)$, $X$, $X^3$ etc
\\
Functions that are products of two even functions are also
\textbf{even} functions.
\newline
Functions that are products of two odd functions are
\textbf{even} functions. (e.g $X \times X^3 = X^4$)
\newline
Functions that are products of an even function and an
odd function are \textbf{odd} functions.
\section{Fourier Series - determining the arguments}
Given a period $2l$, we must determine the form of the fourier
series. $sin( \frac{n x \pi}{l})$
\section{Fourier Series}
X

\newpage
\subsection*{Trigonometric Subsititution}
\[\int\frac{dx}{\sqrt{a^2-x^2}}\]

\[x=a\sin(\theta),\quad dx=a\cos(\theta)\,d\theta, \quad \theta=\arcsin\left(\frac{x}{a}\right)\]

{\large
\begin{eqnarray}
\int\frac{dx}{\sqrt{a^2-x^2}} & = \int\frac{a\cos(\theta)\,d\theta}{\sqrt{a^2-a^2\sin^2(\theta)}}\\ \nonumber &= \int\frac{a\cos(\theta)\,d\theta}{\sqrt{a^2(1-\sin^2(\theta))}} \\ \nonumber
& = \int\frac{a\cos(\theta)\,d\theta}{\sqrt{a^2\cos^2(\theta)}} \\ &= \int d\theta=\theta+C \\ \nonumber &=\arcsin\left(\frac{x}{a}\right)+C
\end{eqnarray}
}
\subsection{Partial Derivatives: Volume of a Cone}

The volume ''V'' of a cone depends on the cone's height ''h'' and its radius 'r' according to the formula
\[V(r, h) = \frac{\pi r^2 h}{3}.\]
The partial derivative of ''V'' with respect to 'r' is
\[\frac{ \partial V}{\partial r} = \frac{ 2 \pi r h}{3},\]

which represents the rate with which a cone's volume changes if its radius is varied and its height is kept constant.
The partial derivative with respect to ''h'' is
\[\frac{ \partial V}{\partial h} = \frac{\pi r^2}{3},\]

which represents the rate with which the volume changes if its height is varied and its radius is kept constant.
\section{Curl}
In vector calculus, the curl is a vector operator that describes the infinitesimal rotation of a 3-dimensional vector field. At every point in the field, the curl of that field is represented by a vector. The attributes of this vector (length and direction) characterize the rotation at that point.
%--------------------------------------------------%
\section{Laplacian Operator}
\[\Delta f = \nabla^2 f = \nabla \cdot \nabla f\]

\subsection*{Trapezoidal Rule}
\[\int_{a}^{b} f(x)\, dx \approx (b-a)\frac{f(a) + f(b)}{2}\]

%-------------------------------%
\subsection*{Laplacian Analysis: convolution}

\[ (f * g)(t) = \int_0^t f(\tau)g(t-\tau)\,d\tau= F(s) \cdot G(s) \]


%--------------------------------------------------%
\section{Notation}
\[ \begin{vmatrix} \mathbf{i} & \mathbf{j} & \mathbf{k} \\
{\frac{\partial}{\partial x}} & {\frac{\partial}{\partial y}} & {\frac{\partial}{\partial z}} \\
 F_x & F_y & F_z \end{vmatrix}\]

\[\left(\frac{\partial F_z}{\partial y}  - \frac{\partial F_y}{\partial z}\right) \mathbf{i} + \left(\frac{\partial F_x}{\partial z} - \frac{\partial F_z}{\partial x}\right) \mathbf{j} + \left(\frac{\partial F_y}{\partial x} - \frac{\partial F_x}{\partial y}\right) \mathbf{k}\]

 %--------------------------------------------------%
\section{Example}
\[f(x,y,z)= \ 2x+3y^2-\sin(z)\]

\[ \nabla f=
\frac{\partial f}{\partial x} \mathbf{i} +
\frac{\partial f}{\partial y} \mathbf{j} +
\frac{\partial f}{\partial z} \mathbf{k}
 = 2\mathbf{i}+ 6y\mathbf{j} -\cos(z)\mathbf{k}\]

\section{Stoke's Theorem}
\[
\iint_{\Sigma} \nabla \times \mathbf{F} \cdot d\mathbf{\Sigma} = \oint_{\partial\Sigma} \mathbf{F} \cdot d \mathbf{r}, \]

\section{Laplace Transforms }
If $g(t)=k \times f(t)$ then $G(S) = k \times F(S)$ where $k$ is a
constant. $\mathcal{f(t)}=F(S)$.
\begin{eqnarray}
f(t) &=& (t+1)^2\\
&=& t^2 +2t +1 \nonumber
\end{eqnarray}


\section{Laplace Transforms Using 1st Shifting Theorem}
\begin{eqnarray}
 g(t) = e^{at}f(t) \quad \Leftrightarrow \quad G(S)= F(S-a) \nonumber
\end{eqnarray}
The function $g(t)$ is presented in a form whereby $a$ and $f(t)$
are easily identifiable. First determine $F(S)$ by finding the
Laplace transform of $f(t)$. Then replace all $S$ terms with
$S-a$.

\section{Laplace Transforms Using 2nd Shifting Theorem}

\begin{eqnarray}
 g(t) = u^{a}f(t-a) \quad \Leftrightarrow \quad G(S)= e^{-aS}F(S) \nonumber
\end{eqnarray}
The function $g(t)$ is presented in a form whereby $a$ and
$f(t-a)$ are easily identifiable. ($U_{a}(t)$ is called the unit
step function). First determine $f(t)$ by replace all $t-a$ terms
in $f(t-a)$ with $t$. Then calculate the laplace transform of
$f(t)$ i.e. $F(S)$. The solutions is in form $G(S)= e^{-aS}F(S)$.

\section{Inverse Laplace Transforms (2 questions) }
Partial fraction expansion is used in questions 4 and 5.
\section{Inverse Laplace Transforms 2}

The denominator has form $S^2 - 2aS + a^2 + k$ which is equivalent
to $(S-a)^2 + k$. Therefore $G(S)$ will have form $F(S-a)$
\newline
The function $G(S)$ may have the form $\frac{S+D}{S^2 +(C+D)S +
CD}$, where C and D are constants. This expression simplifies
$\frac{S+D}{(S+C)(S+D)}$ and again to $\frac{1}{S+C}$. The inverse
laplace transform $g(t)$ can be easily determined.
\section{Convolution}
We are asked to find a function h(t) which is the convolution of
two given functions $f(t)$ and $g(t)$. i.e $h(t)=h*g(t)$.\newline
Importantly $H(S) = F(S)\times G(S)$. We determine the laplace
transforms, F(S) and G(S), and multiply them to determine H(S). We
then find the inverse Laplace transform of H(S) to yield our
solution.
\subsection{Example}
Find $h(t)$ when $h(t) = f*g(t)$, with $f(t)= e^{t}$ and $g(t)=
e^{-t}$.\newline

\begin{eqnarray}
f(t) = e^{t} \quad \Leftrightarrow \quad F(S)= \frac{1}{S-1}
 \nonumber\\
g(t) = e^{-t} \quad \Leftrightarrow \quad G(S)= \frac{1}{S+1}
 \nonumber\\
H(S) = F(S)\times G(S) = \frac{1}{(S-1)(S+1)}
 \nonumber
\end{eqnarray}
\subsection{Example}
Find $h(t)$ when $h(t) = f*g(t)$, with $f(t)= t$ and $g(t)=
t^2$.\newline

\begin{eqnarray}
f(t) = t \quad \Leftrightarrow \quad F(S)= \frac{1}{S^2}
 \nonumber\\
g(t) = t^2 \quad \Leftrightarrow \quad G(S)= \frac{2}{S^3}
 \nonumber\\
H(S) = F(S)\times G(S) = \frac{2}{S^5}
 \nonumber\\
(H(S) \mbox{ is in form }  k\frac{n!}{S^{n+1}} )
 \nonumber
\end{eqnarray}

With $n=4$, $n!= 4! = 24$. Solving for $k$, $k \times n! = 2$.
Therefore $k=\frac{1}{12}$. The solution is $\cal{L}^{-1}[H(S)]$

%--------------------------------------------------%
\newpage
\subsection*{Laplace Transforms}

\[ \mathcal{L}[f(t)] =  \int^{\infty}_0 f(t) e^{-st}dt \]

\begin{eqnarray}
\mathcal{L}[4] &=&  \int^{\infty}_0 t^2 e^{-st}dt \\
&=&  4 \int^{\infty}_0 e^{-st}dt \\
&=&  4 \left[ \frac{e^{-st}}{-s} \right]^{\infty}_0 \\
&=&  4 \left[ \left(\frac{e^{-\infty}}{-s} \right) -  \left(\frac{e^{-0}}{-s} \right)\right]\\
&=&  \frac{4}{s}
\end{eqnarray}
\subsection*{Fourier Series}

\[ f(x) =  {a_o \over 2} + sum^{\infty}_1 \left( a_ncos(nx) + b_nsin(nx) \right) \]

\[ a_0 = \int^{\pi}_{-\pi}f(x) dx  \]


\subsection*{Fourier Series}

\[ f(x) =  {a_o \over 2} + sum^{\infty}_1 \left( a_ncos(nx) + b_nsin(nx) \right) \]

\[ b_n = {1 \over \pi}\int^{\pi}_{-\pi}f(x) sin(nx)dx  \]


\[ b_n = {1 \over \pi} \left[\int^{0}_{-\pi}-\pi sin(nx)dx + \int^{\pi}_{0}\pi sin(nx)dx  \right] \]


\[ b_n = {1 \over \pi}  \left( \left[ \frac{pi}{n}cos(nx) \right]^0_{-\pi} - \left[ \frac{pi}{n}cos(nx) \right]^{\pi}_0 \right) \]

\[ b_n = {\pi \over n\pi} \left( cos(0) - cos(-n\pi) -  cos(n\pi) + cos(0) \right) \]

\[b_n = { \pi \over n\pi} \left( 2 - 2cos(n\pi)  \right) \]

%----------------------------------------%
%--------------------------------------%
\subsection*{Heaviside function}
$u_1(t)$


\[ \left[U_a(t)  - U_b(t) \right] \times f(t) \]

\textbf{\emph{Special Cases:}}
\begin{itemize}
\item $U_0(t) = 1$
\item $U_{\infty} = 0$
\end{itemize}
\section{Stoke's Theorem}
\[
\iint_{\Sigma} \nabla \times \mathbf{F} \cdot d\mathbf{\Sigma} = \oint_{\partial\Sigma} \mathbf{F} \cdot d \mathbf{r}, \]
\section{Curl}
In vector calculus, the curl is a vector operator that describes the infinitesimal rotation of a 3-dimensional vector field. At every point in the field, the curl of that field is represented by a vector. The attributes of this vector (length and direction) characterize the rotation at that point.
%--------------------------------------------------%
\section{Laplacian Operator}
\[\Delta f = \nabla^2 f = \nabla \cdot \nabla f\]

%--------------------------------------------------%
\section{Notation}
\[ \begin{vmatrix} \mathbf{i} & \mathbf{j} & \mathbf{k} \\
{\frac{\partial}{\partial x}} & {\frac{\partial}{\partial y}} & {\frac{\partial}{\partial z}} \\
 F_x & F_y & F_z \end{vmatrix}\]

\[\left(\frac{\partial F_z}{\partial y}  - \frac{\partial F_y}{\partial z}\right) \mathbf{i} + \left(\frac{\partial F_x}{\partial z} - \frac{\partial F_z}{\partial x}\right) \mathbf{j} + \left(\frac{\partial F_y}{\partial x} - \frac{\partial F_x}{\partial y}\right) \mathbf{k}\]

 %--------------------------------------------------%
\section{Example}
\[f(x,y,z)= \ 2x+3y^2-\sin(z)\]

\[ \nabla f=
\frac{\partial f}{\partial x} \mathbf{i} +
\frac{\partial f}{\partial y} \mathbf{j} +
\frac{\partial f}{\partial z} \mathbf{k}
 = 2\mathbf{i}+ 6y\mathbf{j} -\cos(z)\mathbf{k}\]



%-------------------------------%
\subsection*{Newton-Raphson Method}
\[x_{1} = x_0 - \frac{f(x_0)}{f'(x_0)}\]
\[x_{n+1} = x_n - \frac{f(x_n)}{f'(x_n)} \]
\section{Laplace Transforms }
If $g(t)=k \times f(t)$ then $G(S) = k \times F(S)$ where $k$ is a
constant. $\mathcal{f(t)}=F(S)$.
\begin{eqnarray}
f(t) &=& (t+1)^2\\
&=& t^2 +2t +1 \nonumber
\end{eqnarray}

%------------------------------------%

\subsection*{Partial Fraction Expansion}
\begin{itemize}
\item Distinct Linear Factors
\item Repeated Linear Factors
\item Distinct Quadratic Factors
\item Repeated Quadratic Factors
\end{itemize}
%--------------------------------------------------%
\subsection*{Numerical Methods: Syllabus}
\begin{itemize}
\item Numerical Differentiation and Integration Approximation formulae for derivatives. Trapezoidal rule, Simpson’s rule, Use of error estimates.
\item Numerical Linear Algebra Linear least squares approximation. The above algorithms will be used to solve problems in mathematics and science using the Matlab and Derive computer packages.
\item Solving Systems of Linear Equations Gaussian and Gauss/Jordan elimination, error accumulation, introduction to iterative techniques (Jacobi method). LU decomposition.
\item Solution of Non-Linear Equations Bracketing methods, linear interpolation technique, fixed point iteration, the Newton-Raphson method. Error analysis of iterative methods.
\item Mathematical Preliminaries Computer representation of numbers, types of computational error. Condition and stability of numerical algorithms.
\item Interpolation Piecewise-linear interpolation and Lagrange interpolating polynomial.
\end{itemize}

\subsection*{The Secant method }

Use the secant method to evaluate a root for each of the equations on Sheet 1, subject to
the required accuracy restrictions. Compare the secant method with the previous methods
in each case.

%--------------------------------------------------%
\subsection*{Conservative Vector Fields }
A vector field A is called conservative if any of the following equivalent
conditions holds
\begin{itemize}
\item The line integral of A between two points is independent of the path
\item The line integral of A over any closed curve C is equal to zero, that is
\end{itemize}

%The curl of \textbf{A} is zero, i.e. $\del \times \textbf{A} =0$
The exists a scalar field, called the potential, such that
%\[\textbf{A} = \del \phi \]

%--------------------------------------------------%
\subsection*{The divergence theorem}

%--------------------------------------------------%
\subsection*{Stokes’ Theorem}

The line integral of a vector A taken around a simple closed curve (that is,
a non-intersecting closed curve), C, is equal to the surface integral of the
curl of A taken over any surface S having C as a boundary.


\end{document}
