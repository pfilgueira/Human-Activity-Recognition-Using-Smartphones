
%=========================================================%

\section{1.A Part 2 Guidance on hypothesis testing}

For the most part, this module will use the $p-$value approach to interpreting a hypothesis test.

For the sake of simplicity, we will 
\begin{itemize}
\item If the p-value is less than 0.02, we reject the null hypothesis.
\item If the p-value is greater than 0.02, we fail to reject the null hypothesis.
\end{itemize}

Again, the choice of 0.02 as a threshold is arbitrary. Conventionally a p-value between 0.01 and 0.05 would 
indicate that the testing procedure should be re-appraised, rather than being used as a basis for a decision.

However, later on, we will disgress from this approach, using the "star" system, which is directly implemented 
with some statistical procedures.


%=========================================================%

\section{1.A Part 3 Review of the Normal Distribution}


\subsection{Quick Revision}

\subsection{Histograms}

\subsection{Normal Probability Plots}

\subsection{Remarks on some Forthcoming Material}

\subsubsection{Formal Testing}

\subsubsection{Non Parametric Inference}



%=========================================================%

\section{1.A Part 4 How thsi Module is to be graded}

\begin{enumerate}
\item 15% for Laboratory participation
\item 15% for Computer Exam
\item 15% for optional Written Midterm. (You may have the result disregarded after the exam)
\item 70% (55%) for End of year paper. 
\end{enumerate}
