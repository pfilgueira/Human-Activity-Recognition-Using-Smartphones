The level of marginal significance within a statistical hypothesis test, representing the probability of the occurrence of a given event. The p-value is used as an alternative to rejection points to provide the smallest level of significance at which the null hypothesis would be rejected. The smaller the p-value, the stronger the evidence is in favor of the alternative hypothesis.

P-values are calculated using p-value tables, or spreadsheet/statistical software.

%----------------------------------------------------------------------------%

Because different researchers use different levels of significance when examining a question, a reader may sometimes have difficulty comparing results from two different tests. 

For example, if two studies of returns from two particular assets were done using two different significance levels, a reader could not compare the probability of returns for the two assets easily. For ease of comparison, researchers will often feature the p-value in the hypothesis test and allow the reader to interpret the statistical significance themselves. This is called a p-value approach to hypothesis testing. 

%----------------------------------------------------------------------------%

***
*
PAGE RETIRED: Click here for the new StatsDirect help system.
*
OR YOU WILL BE REDIRECTED IN 5 SECONDS
*
***
Show 
P values
 
The P value or calculated probability is the estimated probability of rejecting the null hypothesis (H0) of a study question when that hypothesis is true.
 
The null hypothesis is usually an hypothesis of "no difference" e.g. no difference between blood pressures in group A and group B. Define a null hypothesis for each study question clearly before the start of your study.
 
The only situation in which you should use a one sided P value is when a large change in an unexpected direction would have absolutely no relevance to your study. This situation is unusual; if you are in any doubt then use a two sided P value.
 
The term significance level (alpha) is used to refer to a pre-chosen probability and the term "P value" is used to indicate a probability that you calculate after a given study.
 
The alternative hypothesis (H1) is the opposite of the null hypothesis; in plain language terms this is usually the hypothesis you set out to investigate. For example, question is "is there a significant (not due to chance) difference in blood pressures between groups A and B if we give group A the test drug and group B a sugar pill?" and alternative hypothesis is " there is a difference in blood pressures between groups A and B if we give group A the test drug and group B a sugar pill".
 
If your P value is less than the chosen significance level then you reject the null hypothesis i.e. accept that your sample gives reasonable evidence to support the alternative hypothesis. It does NOT imply a "meaningful" or "important" difference; that is for you to decide when considering the real-world relevance of your result.
 
The choice of significance level at which you reject H0 is arbitrary. Conventionally the 5% (less than 1 in 20 chance of being wrong), 1% and 0.1% (P < 0.05, 0.01 and 0.001) levels have been used. These numbers can give a false sense of security.
 
In the ideal world, we would be able to define a "perfectly" random sample, the most appropriate test and one definitive conclusion. We simply cannot. What we can do is try to optimise all stages of our research to minimise sources of uncertainty. When presenting P values some groups find it helpful to use the asterisk rating system as well as quoting the P value:
P < 0.05 *
P < 0.01 **
P < 0.001
 
Most authors refer to statistically significant as P < 0.05 and statistically highly significant as P < 0.001 (less than one in a thousand chance of being wrong).
 
The asterisk system avoids the woolly term "significant". Please note, however, that many statisticians do not like the asterisk rating system when it is used without showing P values. As a rule of thumb, if you can quote an exact P value then do. You might also want to refer to a quoted exact P value as an asterisk in text narrative or tables of contrasts elsewhere in a report.
 
At this point, a word about error. Type I error is the false rejection of the null hypothesis and type II error is the false acceptance of the null hypothesis. As an aid memoir: think that our cynical society rejects before it accepts.
 
The significance level (alpha) is the probability of type I error. The power of a test is one minus the probability of type II error (beta). Power should be maximised when selecting statistical methods. If you want to estimate sample sizes then you must understand all of the terms mentioned here.
 
The following table shows the relationship between power and error in hypothesis testing:
 
 
DECISION
TRUTH
Accept H0
Reject H0
H0 is true
correct decision P
type I error P
 
1-alpha
alpha (significance)
H0 is false
type II error P
correct decision P
 
beta
1-beta (power)
 
 
 
H0 = null hypothesis
 
 
P = probability
 
 
 
If you are interested in further details of probability and sampling theory at this point then please refer to one of the general texts listed in the reference section.
 
You must understand confidence intervals if you intend to quote P values in reports and papers. Statistical referees of scientific journals expect authors to quote confidence intervals with greater prominence than P values.
 
Notes about Type I error:
is the incorrect rejection of the null hypothesis
maximum probability is set in advance as alpha
is not affected by sample size as it is set in advance
increases with the number of tests or end points (i.e. do 20 tests and 1 is likely to be wrongly significant)
 
Notes about Type II error:
is the incorrect acceptance of the null hypothesis
probability is beta
beta depends upon sample size and alpha
can't be estimated except as a function of the true population effect
beta gets smaller as the sample size gets larger
beta gets smaller as the number of tests or end points increases
 
Copyright © 2000-2011 StatsDirect Limited, all rights reserved.
