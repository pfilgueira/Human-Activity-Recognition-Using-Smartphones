
%----------------------------------------------------------------------% 
% 2000 - Q1
% Part(a) 

Ozone readings (ppm) were taken at noon at Shannon Airport on 16 consecutive days and the results were recorded as follows:

\begin{verbatim} 
10	14	13	18	12	22	14	19
22	13	14	16	  3	  6	  7	  9
\end{verbatim}

\begin{itemize}
\item[(i)]                   Compute the lower quartile, the upper quartile and the interquartile range
\item[(ii)]                Construct a box plot for the ozone readings
\item[(iii)]               Comment on the box plot – are there mild or extreme outlines, is the data symmetrical etc.
\end{itemize}

%------------------------------------------% 
% Part (b)               
A frequency distribution for bus travel times on a non quality bus corridor in Dublin during early morning peak traffic is as follows:
 
Class Interval	&	Frequency	&	Relative Frequency \\ \hline
15-<16		&	 4		&	0.02 \\ \hline
16-<17		&	15		&	0.075 \\ \hline
17-<18		&	26		&	0.13 \\ \hline
18-<19		&	99		&	0.495 \\ \hline
19-<20		&	36		&	0.18 \\ \hline
20-<21		&	 8		&	0.04 \\ \hline
21-<22		&	12		&	0.06 \\ \hline

\begin{itemize}
\item[(i)]                  Use an Ogive * to compute the following precentiles (approx):
10th, 90th, 95th
\item[(ii)]                Calculate the mean and the median times for journey on this route at peak times.
\end{itemize}
Ogive = cumulative relative frequency
 
%----------------------------------------------------------------------%
 
 
%-----------------------------------------------------------------------------% 
% 2001 - Q1 	

A fire insurance company wants to relate the amount of fire damage in major residential fires to the distance between the residence and the nearest fire station. The study is to be conducted in a large suburb of a major city; a sample of 15 recent fires in this suburb is selected. The amount of damage (thousands of pounds) and the distance (in Kilometers) between the fire and the nearest station are recorded for each fire. 
______________________________________________________________________________
Distance (X) 	 3.4   1.8    4.6    2.3    3.1    5.5    0.7    3.0    2.6    4.3    2.1    1.1    6.1    4.8   3.8 
Fire Damage (Y)   21    13     26     18     23     31     9       17     15     26     19     12     38     31    21    
 
(i)   Illustrate the data in a scatter diagram. 					4 
 
(ii)  Compute the value of the correlation coeffcient. 				6 
 
(iii) Using a significance level of a = 0.05, test the claim that there is no linear      correlation  between distance and fire damage. 				 6 
 
(iv)  Find the equation of the regression line and plot the regression line on the scatter  diagram. Interpret the equation. 					5 
 
(v)  Estimate the amount of fire damage if the distance between the fire and the  nearest  fire station is four miles. 						4 
 
Note that 
S x = 49.2 
S y = 320 
S x 2 = 196.16 
S y 2 = 7722 
S xy = 1219.2 
 
%-----------------------------------------------------------------------------% 
% 2001 - Q 2
	
The following table for directions of traffic movements through an intersection in late afternoon has been compiled 
_________________________________________________________________________
Vehicles going to:_______________________________
Vehicles coming from:	B 1 : North 	B 2 : South    B 3 : East     B 4 : West      Total 
A 1 : North 			0 		28 	       6 		   7 	  41 
A 2 : South 			12 		0 	       3 		   3 	  18 
A 3 : East 			 	5 		1 	        0 		   5           11 
A 4 : West 			10 		9 	       11              	    0          30         
Total : 				27 		38 	       20                    15        100 
 	
(a) Construct a joint probability table for the above data. 			6 
 
(b) What is the probability that a vehicle goes north or west. 			4 
 
(c) What is the probability that a vehicle coming from south goes west. 	3 
 
(d) What is the probability that a vehicle going east came from the south. 	4 
 
(e) Determine 
 
  (i) the probability that either A 1 or B 2 occurs. 				2 
 
 (ii) the probability that both A 2 and B 4 occur. 				2 
 
 (f)  What is the difference between mutually exclusive and independent events.  Use data  where appropriate, from the table to justify your answer. 4
 
 
%-----------------------------------------------------------------------------% 
% 2001 - Q 3	

(a) A new printer is fully warranted during its first year. The number of service calls  during the year of warranty is Poisson distributed with a mean of 1.0. 
 
\begin{itemize} 
\item[(i)] What is the probability of no service calls during the warranty period for a printer?                                                                                                            4 
 
\item[(ii)] What is the probability of fewer than 3 service calls? 			4 
 
\item[(iii)] If each service call costs £20, what is the expected cost of warranty service calls per  printer? What is the standard deviation of the cost? What is the probability that the  total cost of service calls for the year will exceed the expected cost? 											4 
 
 
\item[(iv)] If it were discovered that 45\% of printers had no warranty service calls but about 9\%   required 4 or more calls, would this correspond with the results of a Poisson process?  Explain. 								4 
\end{itemize}
 
%----------------------------------------------------------------------% 
(b) A manufacturer of radios claims that 5% of his radios do not meet the quality 
  specification. A shop keeper will accept a shipment if, one lot containing 10 radios, taken at random, contains only 0 or 1 defective ones; otherwise he will return the shipment. 
 
\begin{itemize} 
\item[(i)] What is the probability that he will return the shipment when the  manufacturer's claim  is correct. 						4  

\item[(ii)] What is the probability that he will accept the shipment when there are actually 10\%  defectives? 							5 
\end{itemize}
%-----------------------------------------------------------------------------% 
% 2001 - Question 4 
The strength of dosage of a plant growth enhancement chemical 
is often measured by the proportion of plants that grow faster. 
A particular dosage of the chemical is fed to 115 plants of these plants, 94 actually show faster growth. 

\begin{itemize}       
\item[(a)] Calculate a point estimate $\hat{p}$ for the proportion of plants that grow faster due to the  dosage. 									3 
 
\item[(b)] What is the standard error of the estimate in (a)? 			
 
\item[(c)] Find a 95\% confidence interval for p. 					 
 
\item[(d)] What size sample is required to give the estimate in (c) to within $\pm 1\%$. 	 
 
\item[(e)] Carry out a testing procedure to investigate the claim that the chemical is 90\% effective.

\end{itemize}								7 
 
%-----------------------------------------------------------------------------% 
% 2001 - Q 5 	
The frequency distribution of the mileage travelled before the first major motor failure 
for each of 182 trucks is given below 
 
 

 
Distance Traveled 
(Thousands of Miles) 		Frequency 
0- < 20 			      6 
20- < 40 			      11 
40- < 60 			      15 
60- < 80 			      23 
80- < 100 			      32 
100- < 120 		     43 
120- < 140 	                 32 
140- < 160 	                 16 
160- < 180 	                  2 
180- < 200 	                  2 
 
(a) Draw the histogram corresponding to this frequency distribution. 	4 

(b) Calculate the mean and median for the above data set.			6 
 
(c) Complete the above frequency table with the relative and cumulative relative frequencies. Use an Ogive (cumulative relative frequency) to compute the following  percentiles  (approx): 10 th , 90 th , 95 th . 				7 
 
(d) Calculate the interquartile range. 						5
 
(e) Construct a box plot for the above data. 					3



%----------------------------------------------------------------------% 
% 2000 - Q2
(a)	(a)    An urn contains 10 disks – 6 white and 4 red.  Two disks are selected, without replacement, from the urn.  Calculate the following probabilities:
 
(i)	(i)                  At least one disk is white
(ii)	(ii)                Exactly one disk chosen is white
(iii)	(iii)               Neither disk chosen is red
(iv)	(iv)              At most one disk chosen is red

%----------------------------------------------------------------------% 
 
(b)	(b)   A and B are two events, P(A)  the probability A occurs is 0.4 and P(B) the probability that B occurs is 0.5.  Calculate P(AUB) the probability that either A or B occurs if:
(i)	(i)                  A and B are independent events
(ii)	(ii)                A and B are mutually exclusive events

%----------------------------------------------------------------------% 

(c)	(c)    An aeroplane has a built in redundant computer system.  In this system if computer one fails it is bypassed and the second computer system is automatically activated.  Assuming that the probability of failure of any one of these computers is 0.15 and that the failures of these computers are mutually independent events then:
 
(i)	(i)                  Calculate the possibility that the on board computer system does not fail
(ii)	(ii)                An airline wishes that the probability of its computer system failing should be less than 5 in 10,000 i.e. probability that system does not fail = 0.9995.  How many computers would it require in its system to achieve the 0.9995 target?
 
 
 

%----------------------------------------------------------------------% 
% 2000 - Q3
(a)	(a)    Assume that the length of injected moulded plastic components are normally distributed with a mean of 50mm and a standard deviation of 2mm.  Draw a rough sketch and then calculate corresponding probability for the following measurements occurring on an individual component:
 
(i)	(i)                  Between 50 and 52.4mms
(ii)	(ii)                Less than 47.6 mms
(iii)	(iii)               Between 48.2 and 51.6 mms
(iv)	(iv)              Less than 53.8 mms

%----------------------------------------------------------------------% 
 
(b)	(b)   Assume that Z scores are normally distributed with a mean of Zero and a standard deviation of 1
 
(i) 	      P(0< Z < a) = 0.4192	Find a
(iii)	(iii)               P(- b £ Z < b) = 0.92	Find b
(iv)	(iv)              P(Z £ c) = 0.2389 		Find c
 
(c)	(c)    A physical education teacher wishes to classify her students performance on the high jump into the following grades:
 
A = Excellent, B = Very Good, C = Moderate, D = Poor, E = Fail.
 
Furthermore, she wishes to allocate  A Grades to 10% of the students, B to 25% , C to 30%, D to 15% and the rest as E grades.  Journal articles indicate that her students’ performance on the high jump is approximately normally distributed with a mean of 120 cm and a standard deviation of 10cm.  Assuming that the published data is accurate, calculate the heights at which the teacher should set the high jump in order to achieve her desired distribution of grades.
 
 
 
 
 
 
 

%----------------------------------------------------------------------% 
% 2000 - Q4
(a)	(a)    A communications channel transmits digits 0 and 1.  However, due to interference the digit transmitted is incorrectly received with probability 0.2.  In order to increase accuracy an engineer decides to use majority decoding and transmits 00000 instead of a single 0.  The receiver will assume a 0 was transmitted if 3 or more zeroes are received. 
 
(i)	(i)                  What is the probability that the message will be incorrectly decoded?
(ii)	(ii)                If 5 digits are transmitted, what is the expected (average) number of digits transmitted incorrectly?
(iii)	(iii)               What is the probability that exactly two of the 5 digits will be transmitted incorrectly?
 
%----------------------------------------------------------------------% 

(b)	(b)   Telephone calls arrive at a switchboard at the rate of 40 per hour.  Assume that the telecentre operators take 3 minutes to deal with a customer query.  Calculate the probability of :
 
(i)	(i)                  2 or more calls arriving in any 3 minute period.
(ii)	(ii)                No phone calls arriving in a 3 minute period
(iii)	(iii)               Exactly one phone call arriving in any 3 minute period
(iv)	(iv)              Average and standard deviation of the number of phone calls arriving in a 3 minute period.
 
%----------------------------------------------------------------------% 

(c)	(c)    A power supply unit for a computer is assumed to follow an exponential distribution with a mean life of 1,000 hours.  What is the probability that the unit will:
 
(i)	(i)                  fail in the first 100 hours
(ii)	(ii)                survive more than 800 hours
 
 
 

%----------------------------------------------------------------------% 
% 2000 - Q5
The surface finish of a metal part is thought to be linearly related to the cutting speed of the machine which produces it.  Surface finish is measured on an arbitrary scale from 0 to 20, with 0 being the roughest finish.  The following data have been observed:
 
Surface finish (Y)	4.89	5.95	6.32	6.00	7.70	9.50	9.73	9.50
Speed (RPM) (X)	12	13	14	15	16	17	18	19
 
(i)	(i)                  Draw a scatter diagram to illustrate the data
(ii)	(ii)                Find the value of the correlation coefficient
(iii)	(iii)               Using a significance level of a = 0.05, test the claim that there is no linear correlation between surface finish and speed
(iv)	(iv)              Find the equation of the regression line and plot the regression line on the scatter diagram
(v)	(v)                Estimate the surface finish if the machine speed is set at 17rpm. 
 
 
Note that:
Sy = 	59.59
Sx = 	124
Sy2 = 	469.71
Sx2 = 	2075
Sxy	983.58


%----------------------------------------------------------------------% 
% 2002 - Q1

(a)	In exploratory data analysis it is important to test underlying assumptions about the data.  Describe briefly the role of the following techniques in data analysis and indicate in an outline sketch what each would look like for well behaved data.
(i)	Run sequence plot
(ii)	Log plot
(iii)	Histogram
(iv)	Normal probability plot

%----------------------------------------------------------------------% 

(b)	Outline the steps required to construct a Box Plot and draw box plots for the following situations:
(i)	Clean data, symmetrical, not outlines
(ii)	Skewed data, very large data points as outlines

(c)	Construct a Box Plot for the following set of data and comment on its key features:

1, 3, 7, 8, 12, 2, 9, 12, 14, 38

%----------------------------------------------------------------------% 
% 2002 - Q2

%----------------------------------------------------------------------% 

(a)	Assume that the length of injected moulded plastic components are normally distributed with a mean of 100mm and a standard deviation of 4mm.  Draw a rough sketch and then calculate corresponding probability for the following measurements occurring on an individual component:
 
(i)	Between 100 and 104.4mms
(ii)	Less than 97 mms
(iii)	Between 98.2 and 101.6 mms
(iv)	Less than 103.8 mms

%----------------------------------------------------------------------% 

(b)	Assume that Z scores are normally distributed with a mean of Zero and a
 	standard deviation of 1
 
(i) 	P (0< Z < a)  =  0.1915 			 Find   a 

(ii)            P(- b £ Z < b) =  0.90		            	 Find   b

     	(iii)             P(Z £ c)  =       0.3085 			 Find   c


(c)	Your company produces injected mouldings for rotary arms. The company claims that less than 2% of all product produced is defective. A random sample of 100 parts revealed that 4 parts were defective. 

Using a 5% level of significance is there evidence to close the line down? Justify your decision.

%----------------------------------------------------------------------% 
% 2002 - Q3

(a) 	Two injection moulding machines in separate plants manufacture battery
connectors for the same client company.  The critical parameter is the mean cushion distance.  A random sample of 4 products from each company was tested and the results recorded as follows:

	Factory 1	Factory 2
Sample (mm) Measurements	38,     38.5
39.5,  40	(35.5)  (36.5)
(37)     39
Mean cushion distance	39	37
Standard Deviation	Unknown	Unknown

%------------------------------------------% 
You are required to test the hypothesis that there is no statistical difference between the two populations.  You may assume a 10% probability of a Type 1 error.

(b) 	The sample variance for population one is 0.83 and sample variance for population two is 2.17.  Does the data provide sufficient evidence to support the hypothesis that the populations are homogenous.  Use a 5% level of significance and indicate clearly all calculations made.


%----------------------------------------------------------------------% 
% 2002 - Q4

(a)  	Nine per cent of PCB boards purchased over the Internet from the Far East have some defect.  From a large consignment of boards, 50 are chosen at random.  What is the probability that:
(i)	5 or more boards have some defect?
(ii)	Exactly 3 boards are defective?
(iii)	Less than 2 boards are defective?
(iv)	If more than 8 boards from the 50 were defective what action would you take? (justify)

%------------------------------------------%

(b) 	Flaws occur in a heavy duty industry carpet at the rate of 15 per 10 square metre section.  Calculate the probability that:
(i)	3 or more flaws will occur in a one square metre section
(ii)	Exactly 4 flaws will occur in a 10 metre square section
(iii)	14 or less flaws will occur in a 20 square metre section

(c)	There is a constant probability of 0.10 that the power supply in telecoms network will not start.  You are requested to calculate the probability that the power supply will fail the 5th time it is activated.

%----------------------------------------------------------------------% 
% 2002 - Q5

A production manager wanted to establish if there was a relationship between the number of hours recorded for a manufacturing cell and the number of units produced in the cell.  A random sample of 7 different times and their corresponding number of units produced were recorded as follows:


	X	Y
	Hours	Cell Prod.
1	2	5
2	3	6
3	4	7
4	6	10
5	7	9
6	9	12
7	10	12
	
	xy = 407		x2 = 295		x = 41		y = 61

(a) 	You are required to 
i.	Draw a scattergram and comment on its features
ii.	Find the regression equation and plot the regression equation on scattergram


The data was entered into Minitab and the following outputs were generated

\begin{verbatim}
Predictor	Coef		SE Coef	T	P
Constant	3.4063		0.6041		5.64	0.002
Hours		0.90625	0.09305	9.74	0.000

S= 0.6892	R-Sq = 95.0%		R-Sq(adj) = 94.0%
\end{verbatim}

%------------------------------------------%

(b) 	You are requested to explain how the T-value of 9.74 was calculated and to 
interpret the corresponding P value of 0.000

%------------------------------------------%

(c)	Fill in the blanks from the following table and explain the relationship between F value of 94.85 and the T-value of 9.74 in section (b)

Analysis of Variance
Source			DF		SS		MS		F		P
Regression		1		45.054		45.054		94.85		0.000
Residual Error		5		xx		xxx		
Total			6		47.429

%------------------------------------------%
















