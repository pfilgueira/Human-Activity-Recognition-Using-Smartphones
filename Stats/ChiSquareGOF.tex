%UoLep Stats Section 14
Tests for Goodness of Fit
\begin{itemize}
\item Basic Counting Model
\item Goodness of Fit Statistics
\item Testing with unknown parameters
\item Testing for association in two-way tables
\end{itemize}
%======================================================================%

\section{Pearson's chi-squared test}

Chi Square

\begin{description}
\item[Null Hypothesis]
There is no relationship between the two categorical variables

\item[Alternative Hypothesis]
There is a relationship between the two categorical variables
\end{description}
%======================================================================%

Observed Values

Expected Values (under the null hypothesis)

Are the differences between Observed values and the Expected values
small enough to be due to random error (i.e. null hypothesis is valid)
or too large for the null hypothesis to be feasible?

Expected values for each cell

Row Total Column Total
Overall Total

E-O/E

%======================================================================%
\subsection{Degrees of freedom}

\[df=(r-1)(c-1)\]

\begin{itemize}
\item[r] = number of rows
\item[c] = number of columns
\end{itemize}

2 rows and 3 columns r = 2 c = 3 
df= (2-1)(3-1) = (1)(2) = 2



Pearson's chi-squared test uses a measure of goodness of fit which is the sum of differences between observed and expected outcome frequencies (that is, counts of observations), each squared and divided by the expectation:
\[ \chi^2 = \sum_{i=1}^n {\frac{(O_i - E_i)}{E_i}^2} \]
where:

\begin{itemize}
\item[Oi] = an observed frequency (i.e. count) for bin i
\item[Ei] = an expected (theoretical) frequency for bin i, asserted by the null hypothesis.
\end{itemize}
