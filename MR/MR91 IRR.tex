\documentclass{beamer}

\usepackage{amsmath}
\usepackage{amssymb}

\begin{document}

\begin{frame}
\Huge
\[\mbox{Financial Mathematics}\]
\LARGE
\[\mbox{Internal Rate of Return}\]

\Large
\[\mbox{www.stats-lab.com}\]
\[\mbox{Twitter: @StatsLabDublin}\]

\end{frame}


%------------------------------------------------%
\begin{frame}
\frametitle{Internal Rate of Return}
\Large
Decision Criterion for Internal Rate of Return
\begin{itemize}
\item If the IRR is greater than the cost of capital, accept the project.
\item If the IRR is less than the cost of capital, reject the project.
\end{itemize}

\end{frame}
%------------------------------------------------%
\begin{frame}
\frametitle{Internal Rate of Return}
\Large
\textbf{Example:} If an investment may be given by the sequence of cash flows:

\begin{tabular}{|c|c|}
Year (n) &	Cash flow ($C_n$)\\
0 &	-123400 \\
1 &	36200\\
2 &	54800\\
3 &	48100\\
\end{tabular} 

\end{frame}
%------------------------------------------------%
\begin{frame}
\frametitle{Internal Rate of Return}
Then the IRR r is given by
\[\mathrm{NPV} = -123400+\frac{36200}{(1+r)^1} + \frac{54800}{(1+r)^2} + \frac{48100}{(1+r)^3} = 0.\]
In this case, the answer is $5.96\%$ (in the calculation, that is, r = .0596).
\end{frame}

%------------------------------------------------%


\end{document}

Maths Resource

http://www.csusm.edu/mathlab/documents/M132BusCalcFormulas%20r1-12e.pdf
http://www.math.ubc.ca/~chau/elasticity.pdf
http://www.textbooksonline.tn.nic.in/books/12/std12-bm-em-1.pdf
