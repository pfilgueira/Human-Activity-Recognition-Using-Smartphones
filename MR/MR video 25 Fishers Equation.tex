

\documentclass{beamer}

\usepackage{amsmath}
\usepackage{amssymb}
\usepackage{graphics}

\begin{document}
	%-------------------------------------------------- %
	\begin{frame}
		\bigskip
		{
			\Huge
			\[ \mbox{Financial Mathematics}  \]
			\huge
			\[ \mbox{Fisher's Equation}  \]
		}
		
		{
			\LARGE
			\[ \mbox{www.MathsResource.com}  \]
			%\[ \mbox{Twitter: @StatsLabDublin} \]
		}
	\end{frame}

%--------------------------------------------------
\begin{frame}
\frametitle{Fisher's Equation}
\Large
\begin{itemize}
\item The market rate of return on the 4.25\% UK government bond maturing on 8 March 2050 is 3.81\% per year. \item Let's assume that this can be broken down into a real rate of exactly 2\% and an inflation premium of 1.775\% \\ \textit{(no premium for risk, as government bond is considered to be "risk-free")}:
\end{itemize}
\[ 1.02 \times 1.01775 = (1 + 0.02) \times (1 + 0.01775) = 1.0381 \]

\end{frame}
%--------------------------------------------------
	\begin{frame}
		\frametitle{Fisher's Equation}
This article implies that you can ignore the least significant term in the expansion (0.02 × 0.01775 = 0.00035 or 0.035\%) and just call the nominal rate of return 3.775\%, on the grounds that that is almost the same as 3.81\%.
\end{frame}

%--------------------------------------------------
\begin{frame}
\frametitle{Fisher's Equation}
\Large
\begin{itemize}
\item At a nominal rate of return of 3.81\% pa, the value of the bond is £107.84 per £100 nominal. 
\item At a rate of return of 3.775\% pa, the value is £108.50 per £100 nominal, or 66p more.
\end{itemize}
\end{frame}

%--------------------------------------------------
\begin{frame}
	\frametitle{Fisher's Equation}
	\Large
\begin{itemize}
	\item
The average size of actual transactions in this bond in the market in the final quarter of 2005 was £10 million. 
\item So a difference in price of 66p per £100 translates into a difference of £66,000 per deal.
\end{itemize}
\end{frame}

\end{document}

Maths Resource

http://www.csusm.edu/mathlab/documents/M132BusCalcFormulas%20r1-12e.pdf
http://www.math.ubc.ca/~chau/elasticity.pdf
http://www.textbooksonline.tn.nic.in/books/12/std12-bm-em-1.pdf
