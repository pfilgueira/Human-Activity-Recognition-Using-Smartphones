\begin{frame}
\frametitle{Logistic map}
An example of a recurrence relation is the logistic map:
\[x_{n+1} = r x_n (1 - x_n),\]
with a given constant r; given the initial term x0 each subsequent term is determined by this relation.
Some simply defined recurrence relations can have very complex (chaotic) behaviours, and they are a part 
of the field of mathematics known as nonlinear analysis.
Solving a recurrence relation means obtaining a closed-form solution: a non-recursive function of n.
%============================================================================%
\begin{frame}
\frametitle{Fibonacci numbers}
The Fibonacci numbers are the archetype of a linear, homogeneous recurrence relation with constant coefficients (see below). They are defined using the linear recurrence relation
F_n = F_{n-1}+F_{n-2}
with seed values:
F_0 = 0
F_1 = 1
Explicitly, recurrence yields the equations:
\begin{eqnarray}
F_2 = F_1 + F_0
F_3 = F_2 + F_1
F_4 = F_3 + F_2
\ldots
\end{eqnarray}
\end{frame}
%======================================================================%
\begin{frame}
We obtain the sequence of Fibonacci numbers which begins:
\[0, 1, 1, 2, 3, 5, 8, 13, 21, 34, 55, 89, ...\]
It can be solved by methods described below yielding the closed-form expression which involve powers of 
the two roots of the characteristic 
polynomial t2 = t + 1; the generating function of the sequence is the rational function
\[\frac{t}{1-t-t^2}.\]
%============================================================================%
\begin{frame}
\frametitle{Binomial coefficients}
A simple example of a multidimensional recurrence relation is given by the binomial coefficients \tbinom{n}{i}, which count the number of ways of selecting i out of a set of n elements. They can be computed by the recurrence relation
\binom{n}{i}=\binom{n-1}{i-1}+\binom{n-1}{i},
with the base cases \tbinom{n}{0}=\tbinom{n}{n}=1. Using this formula to compute the values of all binomial coefficients generates an infinite array called Pascal's triangle. The same values can also be computed directly by a different formula that is not a recurrence, but that requires multiplication and not just addition to compute: 
\binom{n}{i}=\frac{n!}{i!(n-i)!}.
