\documentclass{beamer}

\usepackage{amsmath}
\usepackage{amssymb}

\begin{document}


\begin{frame}
\Large
\[\mbox{Marginal Cost}\]
\end{frame}

% http://www.bized.co.uk/learn/economics/maths/index.htm?page=16

\begin{frame}
\frametitle{Marginal Cost}

Take the total cost function \[TC = 5q^2 + 8q + 12\]. Calculate the marginal cost for the values q = 5 and q = 8.
\end{frame}
\begin{frame}
\frametitle{Marginal Cost}
We know that marginal cost is the change in total cost as a result of a small change in output. When differentiating the Total Cost curve therefore we are attempting to find the change in total cost as a result of an infinitessimally small change in one or more of the factors affecting total cost. 
\end{frame}
\begin{frame}
\frametitle{Marginal Cost}
Marginal Cost (MC) is the derivative of total cost with respect to output (dTC/dq). 

Given the function, the rule we must use is the Power Function rule (dy/dx = anxn-1,). In this example MC = dTC/dq = 10q + 8.
If q = 5, therefore, MC = 10(5) + 8 = 58
If q = 8, MC = 10(8) + 8 = 88
\end{frame}


\end{document}
