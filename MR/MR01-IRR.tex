%=====================================================================%
\begin{frame}
\Large
{
\Huge
\[ \mbox{ Financial Mathematics } \]
}
{
\LARGE
\[ \mbox{ Internal Rate of Return } \]
}
\bigskip
{
\Large
\[ \mbox{ MathsResource.com } \]
}
\end{frame}
%=====================================================================%
\begin{frame}
\frametitle{Internal Rate of Return}
\Large

The \textbf{Internal Rate of Return} is the interest rate that makes the Net Present Value zero.


\end{frame}
%=====================================================================%
\begin{frame}
\frametitle{Internal Rate of Return}
\Large

Decision Criterion for Internal Rate of Return
\begin{itemize}
\item If the IRR is greater than the cost of capital,
accept the project.
\item If the IRR is less than the cost of capital, reject
the project.
\end{itemize}

\end{frame}
%=====================================================================%
\begin{frame}
\frametitle{Internal Rate of Return}
\Large

Example: If an investment may be given by the
sequence of cash 
ows:
\begin{center}
\begin{tabular}{|c|c|}
Year (n) & Cashflow (Cn) \\ \hline
0 & -123400 \\
1 & 36200 \\
2 & 54800 \\
3 & 48100 \\ \hline
\end{tabular}
\end{center}

\end{frame}
%=====================================================================%
\begin{frame}
\frametitle{Internal Rate of Return}
\Large


Then the IRR, which we will denote below as ``r" , is given by
\[ NPV = -123400 + \frac{123400}{}  + \frac{36200}{(1+r)^1} + \frac{54800}{(1 + r )^2} = \frac{48100}{(1 + r )^3} = 0:
\]

In this case, the answer is 5.96% (in the calculation, that is, r =
0.0596).

\end{frame}
%=====================================================================%
\end{document}
