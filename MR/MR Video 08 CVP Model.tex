\documentclass{beamer}

\usepackage{amsmath}
\usepackage{amssymb}

\begin{document}


\begin{frame}
The assumptions of the CVP model yield the following linear equations for total costs and total revenue (sales):
\text{Total costs} = \text{fixed costs} + (\text{unit variable cost} \times \text{number of units})
\text{Total revenue} = \text{sales price} \times \text{r of units}
\end{frame}

%------------------------------------------------%
\begin{frame}
\frametitle{CVP Model}
These are linear because of the assumptions of constant costs and prices, and there is no distinction between units produced and units sold, as these are assumed to be equal. Note that when such a chart is drawn, the linear CVP model is assumed, often implicitly.
\end{frame}

%------------------------------------------------%
\begin{frame}
\frametitle{CVP Model}
In symbols:
\text{TC} = \text{TFC} + \text{V} \times \text{X}
\text{TR} = \text{P} \times \text{X}
where
TC = Total costs
TFC = Total fixed costs
V = Unit variable cost (variable cost per unit)
X = Number of units
TR = S = Total revenue = Sales
P = (Unit) sales price
Profit is computed as TR-TC; it is a profit if positive, a loss if negative.
\end{frame}

%------------------------------------------------%
\begin{frame}
\frametitle{CVP Model}

\begin{tabular}{|c|c|}
$E_d = 0$ & Perfectly inelastic demand \\
$- 1 < E_d < 0 $ &Inelastic or relatively inelastic demand\\
$E_d=  - 1 $& Unit elastic\\
$ - \infty < E_d < - 1$ & 
Elastic or relatively elastic demand \\
$E_d = \infty $&
Perfectly elastic demand \\
\end{tabular} 
\end{frame}


\begin{frame}
\[ \epsilon = \frac{\operatorname d Q/Q}{\operatorname d P/P} \]



 Revenue is simply the product of unit price times quantity:
  \[ \mbox{Revenue} = PQ_d\]
\end{frame}
\end{document}

Maths Resource

http://www.csusm.edu/mathlab/documents/M132BusCalcFormulas%20r1-12e.pdf
http://www.math.ubc.ca/~chau/elasticity.pdf
http://www.textbooksonline.tn.nic.in/books/12/std12-bm-em-1.pdf
