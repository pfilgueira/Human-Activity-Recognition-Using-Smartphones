\documentclass[12pt]{article}

%opening
\title{Financial Mathematics}
\author{www.Stats-Lab.com}

\begin{document}

\maketitle
%------------------------------ %


\section{Overview}
\begin{enumerate}
\item Key Mathematical Concepts,
\item Simple and Compound Interest,
\item Mathematics for Economics,
\item Investment Appraisal,
\item Annuities and Perpetuities,
\item Price Indices,
\item Time Series Analysis,
\item Derivatives,
\item Fixed Income Instruments.
\end{enumerate}

\section*{2. Simple and Compound Interest}
\begin{enumerate}
\item Simple Interest
\item Compound Interest
\end{enumerate}

\section*{3. Mathematics of Economics}
\begin{enumerate}
\item Price Elasticity of Demand
\end{enumerate}

\section*{4. Simple and Compound Interest}
\begin{enumerate}
\item Net Present Value
\item Internal Rate of Return
\item Weighted Accounting Cost of Capital
\end{enumerate}



Based on modules 

MA4505 (Equine Stats)

MA4603 (Science Maths 3)

MA4004 (Engineering Stats)


Probability
◦
Axioms and Basic Laws

◦
Relationship to proportions



Basic Descriptive statistics
◦
Measures of centrality. Why use the median instead of the mean.

◦
Measures of dispersion.  exposition of the concept of variance.

◦
Skewed Data and Outliers. Median and IQR.



Discrete probability distributions
◦
Binomial ( From formula and from tables)

◦
Poisson (From formula and from tables)

◦
parameters (mean and variance)

◦
Using the poisson to estimate binomial



The Normal distribution 

This one may class with other modules/ topic lists
◦
The Standard Normal (“Z”) Distribution- fully tabulated in all statistical log tables

◦
Using the Z tables (to compute p-values and Z values as necessary)

◦
Relationship between any general normal distribution and the Z distribution (i.e. The standardisation formula)

◦
Three rules : Interval, Symmetry , Complement Rule

◦
Relevance to Hypothesis testing (The Central Limit theorem)



Other Continuous distributions (used in module MA4603)

    1) Uniform

    2) Exponential


Chi Square Goodness of fit (used in modules MA4004 and MA4505)
◦
Relationship between two categorical variables

◦
Expected values when independence is assumed

◦
Underlying theory: Is the difference between Observed values and Expected values ( under Ind. assptn. )due to random effects or due to a the fact that the assumption of     independence is wrong.

◦
determining the Test statistic and critical value. Employing the decision rule




Confidence intervals
◦
expressing accuracy of sample values

◦
interval width

◦
sample mean

◦
quantile (critical value)

◦
Standard Error



Hypothesis testing
◦
The null and alternative hypothesis

◦
Sample size and number of tails

◦
Using the Student t tables

◦
computing the test statistics and critical value

◦
decision rule

◦
p-value approach



Regression models
◦
Simple Linear Regression.

◦
Parameter values and estimate values for slope and intercept

◦
Using the regression equation to make estimates

◦
Residuals

◦
Extrapolation Fallacy ( i.e. dont use outside the sample range)



Correlation
◦
Pearson’s Correlation coefficient.

◦
Linear relatonships

◦
Spurious Correlation. The relationship (or lack thereof) with Causation.




    


\end{document}
