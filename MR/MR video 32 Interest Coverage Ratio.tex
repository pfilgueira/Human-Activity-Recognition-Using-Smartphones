\documentclass{beamer}

\usepackage{amsmath}
\usepackage{amssymb}

\begin{document}


\begin{frame}
\Large
\[
\mbox{Interest Coverage Ratio}
\]
\end{frame}

%---------------------------------------------------------%
\begin{frame}
\frametitle{Interest Coverage Ratio}
A ratio used to determine how easily a company can pay interest on outstanding debt. 
The interest coverage ratio is calculated by dividing a company's earnings 
before interest and taxes (EBIT) of one period by the company's interest expenses of 
the same period:


\[\mbox{Interest Coverage Ratio} = \frac{\mbox{EBIT}}{\mbox{Interest Expense}} \]
 
The lower the ratio, the more the company is burdened by debt expense. 
When a company's interest coverage ratio is 1.5 or lower, its ability to meet interest expenses may be questionable. An interest coverage ratio below 1 indicates the company is not generating sufficient revenues to satisfy interest expenses.

\end{frame}
%---------------------------------------------------------%
\begin{frame}
\frametitle{Interest Coverage Ratio}
Things to Remember 
\begin{itemize}
\item A ratio under 1 means that the company is having problems generating enough cash flow to pay its interest expenses.
\item Ideally you want the ratio to be over 1.5.
\end{itemize}

 A company that barely manages to cover its interest costs may easily fall into bankruptcy if its earnings suffer for even a single month.  
\end{frame}



\end{document}

Maths Resource

http://www.csusm.edu/mathlab/documents/M132BusCalcFormulas%20r1-12e.pdf
http://www.math.ubc.ca/~chau/elasticity.pdf
http://www.textbooksonline.tn.nic.in/books/12/std12-bm-em-1.pdf
