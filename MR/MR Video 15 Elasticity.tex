\documentclass{beamer}

\usepackage{amsmath}
\usepackage{amssymb}

\begin{document}


\begin{frame}
\Huge
\[\mbox{Financial Mathematics}\]
\LARGE
\[\mbox{Price Elasticity of Demand}\]

\Large
\[\mbox{www.MathsResource.com}\]
% \[\mbox{Twitter: @StatsLabDublin}\]
\end{frame}

%------------------------------------------------%
\begin{frame}
\frametitle{Price Elasticity of Demand}
\Large
The Price Elasticity of Demand ($\epsilon$) is computed as follows:

{
	\LARGE
\[ \epsilon = \frac{\operatorname d Q/Q}{\operatorname d P/P} \]
}
\end{frame}

%------------------------------------------------%
\begin{frame}
\frametitle{Price Elasticity of Demand}
\Large
\begin{center}
\begin{tabular}{|c|l|} \hline
$E_d = 0$ & Perfectly inelastic  \\ & demand \\ \hline
$- 1 < E_d < 0 $ & Inelastic or relatively \\ 

& inelastic demand\\ \hline
$E_d=  - 1 $ & Unit elastic\\

 & \\  \hline
$ - \infty < E_d < - 1$ & 
Elastic or relatively\\  & elastic demand \\ \hline 
$E_d = \infty $&
Perfectly elastic demand \\ 

&  \\  \hline
\end{tabular} 
\end{center}
\end{frame}
%------------------------------------------------%
\begin{frame}
\frametitle{Price Elasticity of Demand}
\Large


\textbf{Example 1:} Given x  = f(p) = -2p+15 , determine if demand is elastic, inelastic or unitary 
when p = 4.

Recall: 

$f(4) = -2(4) + 15 = 7$
$f^{\prime}(p) = -2$

$f^{\prime}(4) = -2$


\end{frame}
%------------------------------------------------%
\begin{frame}
	\frametitle{Price Elasticity of Demand}
	\Large
\[ \epsilon = -\frac{4 \times -2}{7} = {8 \over 7} \]


\[ \epsilon \geq 1 \]
\end{frame}
\end{document}

Maths Resource

http://www.csusm.edu/mathlab/documents/M132BusCalcFormulas%20r1-12e.pdf
http://www.math.ubc.ca/~chau/elasticity.pdf
http://www.textbooksonline.tn.nic.in/books/12/std12-bm-em-1.pdf
