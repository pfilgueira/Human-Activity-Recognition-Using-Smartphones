\documentclass{beamer}

\usepackage{amsmath}
\usepackage{amssymb}
\usepackage{graphics}

\begin{document}
\begin{frame}


{
\huge
\[ \mbox{Calculus} \]
\[ \mbox{De Moivre's Theorem} \]
}
{
\Large
\[ \mbox{www.MathsResource.com} \]
}
\end{frame}

%----------------------------------------------------------%
\begin{frame}
\frametitle{De Moivre's Theorem}
\Large
\vspace{-2cm}
Use De Moivre's Theorem to evaluate the following expression
\[\left( cos \frac{\pi}{6} + i\, sin\frac{\pi}{6} \right)^8\]

\end{frame}

%-----------------------------------------------------------%
\begin{frame}
\frametitle{De Moivre's Theorem}
\Large
\vspace{-2cm}
\textbf{De Moivre's Theorem}
{
	\LARGE
\[(Cos\; \theta + i Sin\; \theta)^n = cos\;(n\theta) + i \;sin (n \theta) \]
}
\end{frame}



%-----------------------------------------------------------%
\begin{frame}
\frametitle{De Moivre's Theorem}
\LARGE
\[ \left( cos \frac{\pi}{6} + i\, sin\frac{\pi}{6} \right) ^8\]

\[ = \left[ cos \left(8 \times \frac{\pi}{6} \right) + i\, sin \left(8 \times \frac{\pi}{6} \right) \right] \]

\[ = \left[ cos \left(\frac{4\pi}{3} \right)+ i\, sin \left(\frac{4\pi}{3} \right)\right] \]
\end{frame}

\end{document}
%-----------------------------------------------------------%
\begin{frame}
\frametitle{De Moivre's Theorem}
\Large

\[\left( Cos \left(\frac{4\pi}{3} \right)+ i Sin \left(\frac{4\pi}{3} \right) = -\frac{1}{\sqrt{2}} - i\frac{\sqrt{3}}{2} \]
\end{frame}
%-----------------------------------------------------------%
\begin{frame}
\frametitle{De Moivre's Theorem}
\Large

\end{frame}
%-----------------------------------------------------------%
\begin{frame}
\frametitle{De Moivre's Theorem}
\Large

\end{frame}
\end{document}
