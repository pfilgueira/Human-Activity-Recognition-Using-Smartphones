\documentclass{beamer}

\usepackage{amsmath}

% http://roadtoinvestmentmanagement.blogspot.ie/2012/01/gross-redemption-yield.html

%http://www.finance-investment-business-glossary.com/definitions/gross_redemption_yield_gry.shtml

\begin{document}
\begin{frame}
\frametitle{Gross Redemption Yield}
\begin{itemize}
\item The normal yield value quoted for a bond is the flat yield which is the fixed coupon divided by the current bond price as a percentage. 

\item The Gross Redemption Yield (GRY) takes into account not only the flat yield but also any capital gain received by holding the bond to maturity.
\end{itemize}
\end{frame}
\begin{frame}
\frametitle{Gross Redemption Yield}
\begin{itemize}
\item If for example a bond has a maturity value of £100 in 2 years time but is bought now for £95, there will be a capital gain of 5\% over two years (approxiamtely 2.47\% per annum) to add to the flat yield, if the bond is held to maturity. 
\item If the flat yield is 6\% per annum then the Gross Redemption Yield is 8.47\% per annum.
\end{itemize}
\end{frame}


\end{document}