\begin{document}
% http://www.math.hmc.edu/calculus/tutorials/taylors_thm/
Taylor's Theorem
Suppose we're working with a function f(x) that is continuous and has n+1 continuous derivatives on an interval about x=0. We can approximate f near 0 by a polynomial Pn(x) of degree n:

\begin{itemize}
\item For n=0, the best constant approximation near 0 is
P0(x)=f(0)
which matches f at 0. 
\item 
For n=1, the best linear approximation near 0 is
P1(x)=f(0)+f(0)x
Note that P1 matches f at 0 and P1 matches f at 0. 
\item 
For n=2, the best quadratic approximation near 0 is
P2(x)=f(0)+f(0)x+2!f(0)x2
Note that P2, P2, and P2 match f, f, and f, respectively, at 0.
Continuing this process,

Pn(x)=f(0)+f(0)x+2!f(x)x2++n!f(n)(0)xn

This is the Taylor polynomial of degree n about 0 (also called the Maclaurin series of degree n). More generally, if f has n+1 continuous derivatives at x=a, the Taylor series of degree n about a is
nk=0k!f(k)(a)(x−a)k=f(a)+f(a)(x−a)+2!f(a)(x−a)2++n!f(n)(a)(x−a)n 
This formula approximates f(x) near a. Taylor's Theorem gives bounds for the error in this approximation:

\section{Taylor's Theorem}

Suppose f has n+1 continuous derivatives on an open interval containing a. Then for each x in the interval,
f(x)=nk=0k!f(k)(a)(x−a)k+Rn+1(x) 
where the error term Rn+1(x) satisfies Rn+1(x)=f(n+1)(c)(n+1)!(x−a)n+1 for some c between a and x.

This form for the error Rn+1(x), derived in 1797 by Joseph Lagrange, is called the Lagrange formula for the remainder. The infinite Taylor series converges to f,
f(x)=k=0k!f(k)(a)(x−a)k 
if and only if limnRn(x)=0.
\end{frame}
%==================================================================================%
\begin{frame}
\frametitle{Taylor Series}
\Large

Examples of Taylor Series about 0

For f(x)=ex,
f(k)(x)=ex=f(k)(0)=1
So
ex = = 1+x+2!x2+3!x3+ k=0k!xk  
which converges for all x since limnRn(x)=limnecx(n+1)(n+1)!=0 for all c between 0 and x. 

For f(x)=ln(1+x),
f(x)=ln(1+x) f(x)=11+x f(x)=−1(1+x)2 f(x)=2(1+x)3 f(4)(x)=−32(1+x)4                             =                      f(0)=0 f(0)=1 f(0)=−1 f(0)=2 f(4)(x)=−6     
So
ln(1+x) = = x−2x2+3x3−4x4+ k=0(−1)kxk+1k+1  
which converges only for $-1

\end{frame}
%==================================================================================%
\begin{frame}
\frametitle{Taylor Series}
\Large

The Taylor Series in (x−a) is the unique power series in (x−a) converging to f(x) on an interval containing a. For this reason,

By Example 1,
e−2x=1−2x+2x2−34x3+
where we have substituted −2x for x. 

By Example 2, since ddx[ln(1+x)]=11+x, we can differentiate the Taylor series for ln(1+x) to obtain
11+x=1−x+x2−x3+
Substituting −x for x,
11−x=1+x+x2+x3+

%-----------------------------------------------------------%
\end{document}
