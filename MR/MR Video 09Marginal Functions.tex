\documentclass{beamer}

\usepackage{amsmath}
\usepackage{amssymb}

\begin{document}


\begin{frame}
\Huge
\[\mbox{Financial Mathematics}\]
\LARGE
\[\mbox{Marginal Cost Function}\]

\Large
\[\mbox{www.stats-lab.com}\]
\[\mbox{Twitter: @StatsLabDublin}\]

\end{frame}
%-------------------------------------------------------- %

\begin{frame}
\frametitle{Marginal Cost Function}
\Large
Suppose we have a total cost function $TC(q)$ given as follows:
\[TC(q) = 60000 + 24q + 0.002q^2\]
Answer the following questions:
\begin{enumerate}
\item Find the fixed cost ($FC$) and the marginal cost ($MC$).
\item What is the marginal cost when the output is 200 units?
\item What is the marginal cost when the output is 5000 units?
\end{enumerate}
\end{frame}
%--------------------------------------------------------- %
\begin{frame}
\frametitle{Marginal Cost Function}
\Large
\textbf{Comput the fixed cost ($FC$) }\\
\bigskip
The fixed cost is determined by letting $q=0$

\[TC(q=0) = 60,000 + 24(0) + 0.002(0)^2 \]

\phantomsection{Necessarily the fixed cost is 60,000.}
\end{frame}
%--------------------------------------------------------- %
\begin{frame}
\frametitle{Marginal Cost Function}
\Large
\textbf{Comput the Fixed Cost ($FC$) }\\
\bigskip
The fixed cost is determined by letting $q=0$

\[TC(q=0) = 60000 + 24(0) + 0.002(0)^2 \]

Necessarily the fixed cost is \textbf{60,000}.
\end{frame}

%--------------------------------------------------------- %

\begin{frame}
\frametitle{Marginal Cost Function}
\Large
\vspace{-1.5cm}
\textbf{Compute the Marginal Cost ($MC$) }\\
\bigskip
The marginal cost is determined by differentiating the total cost function (i.e. $TC(q)$) with respect to the variable $q$.
\[TC(q) = 60000 + 24q + 0.002q^2\]
\[MC(q) = \frac{d(TC)}{dq}\]

\end{frame}
%--------------------------------------------------------- %

\begin{frame}
\frametitle{Marginal Cost Function}
\Large
\vspace{-1.8cm}
\textbf{Compute the Marginal Cost ($MC$) }\\
\bigskip
The marginal cost is determined by differentiating the total cost function (i.e. $TC(q)$) with respect to the variable $q$.
\LARGE
\[MC =  24 + 0.004q\]

\end{frame}
\end{document}