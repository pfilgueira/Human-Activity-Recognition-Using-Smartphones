
Sample Space

[http://cnx.org/content/m16845/latest/]

A complete list of all possible outcomes of a random experiment is called sample space or possibility space and is denoted by S.


A sample space is a set or collection of outcome of a particular random experiment.


For example, imagine a dart board. You are trying to find the probability of getting a bullseye. The dart board is the sample space. The probability of a dart hitting the dart board is 1.0. For another example, imagine rolling a six sided die. The sample space is {1, 2, 3, 4, 5, 6}.


The following list consists of sample spaces of examples of random experiments and their respective outcomes.

The tossing of a coin, sample space is {Heads, Tails}

The roll of a die, sample space is {1, 2, 3, 4, 5, 6}

The selection of a numbered ball (1-50) in an urn, sample space is {1, 2, 3, 4, 5, ...., 50}

Percentage of calls dropped due to errors over a particular time period, sample space is {2%, 14%, 23%, ......}

The time difference between two messages arriving at a message centre, sample space is {0, ...., infinity}

The time difference between two different voice calls over a particular network, sample space is {0, ...., infinity}

