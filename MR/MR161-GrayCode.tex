%===========================================================================%
\begin{frame}
\frametitle{Gray Code}
\Large

The reflected binary code, also known as Gray code after Frank Gray, is a binary numeral 
system where two successive values differ in only one bit (binary digit). The reflected binary code was originally designed to prevent spurious output from electromechanical switches. Today, Gray codes are widely used to facilitate error correction in digital communications such as digital terrestrial 
television and some cable TV systems.
\end{frame}

%===========================================================================%
\begin{frame}
\frametitle{Gray Code}
\Large

Dec  Gray   Binary
 0   000    000
 1   001    001
 2   011    010
 3   010    011
 4   110    100
 5   111    101
 6   101    110
 7   100    111

\end{frame}
%===========================================================================%
The Gray code is a type of cyclical binary code patented for the first time in 1947, but not given the name Gray code until the early 1950s, in subsequent patent applications. Specifically, the Gray code is a reflected binary code, meaning that the last numbers in the string can be the same as the beginning numbers, but in reverse order, thus allowing for building on and expanding the usefulness of standard or natural binary code. Frank Gray, the Bell Labs researcher for whom the code is named, developed this particular binary numeral system to help control electromechanical switches. Today, Gray code is used in a variety of environments, particularly digital communications where analog signals must be converted to digital mediums.

During the early stages of development for the Gray code, the focus was primarily on the more effective operation of electromechanical switches. Mechanical switches using natural binary code can be difficult to read in terms of position. Several switches can change position at one time, with complicated transitional positions. Depending on the phase of transition, a switch might read in one position when it is actually in a state of transition, on its way to another position. Multiplied by several switches, an incorrect reading of switch position can result in system-wide errors and false information.

Alternatively, only one switch changes position at a time when using Gray code, which eliminates the possibility of false or misleading position information, since only one bit changes at a time. As development of the Gray code continued for several decades after Gray's initial introduction, additional applications came to light. For example, rotary and optical encoders use Gray code because each sequence or change in position differs by only one bit. Likewise, error correction for digital communication, genetic algorithms, and certain types of maps use Gray code, also due to the single bit change property associated with the code.

A similar reflected binary code was used in the late 1800s in telegraphy. Even earlier, mathematicians used reflected binary code to solve complex mathematical questions or puzzles similar to the Tower of Hanoi and ancient Chinese ring puzzles. Although used, such binary codes were not standardized, patented, or otherwise deemed of proprietary use until Gray's work in the late 1940s and early 1950s. Bell Labs, in using a vacuum tube instrument invented by Frank Gray, patented the first device in which analog signals were converted to reflected binary code. In the 1953 patent application for the device, known as pulse code communication tube or PCM tube, Bell Labs referenced Gray code officially for the first time.
