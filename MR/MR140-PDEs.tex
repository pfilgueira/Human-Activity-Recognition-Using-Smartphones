Partial Differential Equations

Classification 
hyperbolic
parabolic
elliptical
%=========================================================================%
\begin{frame}
Boundary Value Problems
Wave Equation
Heat Equation


\end{frame}
%=========================================================================%
\begin{frame}
\frametitle{Partial Differential Equations}


Partial Differential Equations

\end{frame}
%=========================================================================%
\begin{frame}
\frametitle{Partial Differential Equations}


 

Laplacian operator

 

=2x2

=2x2+2y2

\end{frame}
%=========================================================================%
\begin{frame}
\frametitle{Partial Differential Equations}Laplace Equation


2= 0


2 is the Laplace operator and  is a scalar function.


\end{frame}
%=========================================================================%
\begin{frame}
\frametitle{Partial Differential Equations}


The Heat Equation


\end{frame}
%=========================================================================%
\begin{frame}
\frametitle{Partial Differential Equations}
The Wave Equation
\end{frame}
%=========================================================================%
\begin{frame}
\frametitle{Partial Differential Equations}
Parabolic PDEs

A parabolic partial differential equation is a type of second-order partial differential equation (PDE), describing a wide family of problems in science including heat diffusion, ocean acoustic propagation, and stock option pricing. These problems, also known as evolution problems, describe physical or mathematical systems with a time variable, and which behave essentially like heat diffusing through a solid.


A partial differential equation of the form    Auxx+Buxy+Cuyy+= 0


is parabolic if it satisfies the condition    B2- 4AC = 0.

\end{frame}
%=========================================================================%
\begin{frame}
\frametitle{Partial Differential Equations}
Elliptical PDEs


Hyperbolic


