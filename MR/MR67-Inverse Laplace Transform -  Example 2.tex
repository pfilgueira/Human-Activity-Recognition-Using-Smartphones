
\documentclass{beamer}
\title{A Tiny Example}
\author{Kevin O'Brien}
\date{June 15, 2005}
\begin{document}
\maketitle
%---------------------------------------------------------------%
\begin{frame}
\frametitle{Inverse Laplace Transforms} 

Compute the inverse Laplace transform of the following expression:
{
\large
\[ F(s) = \frac{s+3}{s^2+3s+2} \]

}\end{frame}
%---------------------------------------------------------------%
\begin{frame}
\frametitle{Inverse Laplace Transforms} 
{
\large
\[ F(s) = \frac{s+3}{s^2+3s+2} \]
}
Step 1: Factorize the denominator:

{
\large
\[s^2+3s+2 = (s+2)(s+1)\]
}

\end{frame}
%----------------------------------
\begin{frame}
\frametitle{Inverse Laplace Transforms} 
Re-expressing the function:
{
\large
\[ F(s) = \frac{s+3}{(s+2)(s+1)} \]
}

Split into two separate terms:
{
\large
\[ F(s) = \frac{s}{(s+2)(s+1)} + \frac{3}{(s+2)(s+1)} \]
}

\end{frame}
%----------------------------------
\begin{frame}
\frametitle{Inverse Laplace Transforms} 
\vspace{-0.5cm}
Re-expressing the function:
{
\large
\[ F(s) = \frac{s+3}{(s+2)(s+1)} \]
}
Split into two separate terms:
{
\large
\[ F(s) = \frac{s}{(s+2)(s+1)} + \frac{3}{(s+2)(s+1)} \]
}
\\
\vspace{1cm}
(We will call these terms $F_1(s)$ and $F_2(s)$ respectively. )
\end{frame}
\end{document}
