
 

Todays Class

 
•
We finished Section 1 (introductory Theory ) yesterday


 
•
We also started the first topic of Section 2 : graphical methods


 
•
Today we will look at

◦
Describing qualitative data

◦
Describing quantitative data

◦
Relative frequency tables

◦
Cumulative relative frequencies


 
•
Numerical methods

◦
Today we will look at measures of centrality, such as the mean and median.


 

 



 The type of graphical method we use to describe our data depends on whether the data is quantitative or qualitative.

 

Describing Qualitative Data

 

This type of data does not have numerical values. The data can be divided into categories and the number of cases falling into each category can be counted. The number of cases in each category is called the frequency.

 

Table 2.1 shows the frequency of individuals with different levels of education in a particular workforce. The relative frequency expresses the frequency counts as a percentage of the total number of cases. We call Table 2.1 a frequency distribution.

 

Table 2.1

 

Education Level        Frequency      Relative Frequency

Primary School           432                 56.25 %

Secondary school       284                 36.98 %  

Third Level                 52                   6.77 %

Total                        768                100.00 %

  

 

 2.1.2 Describing Quantitative Data

 

Quantitative data have numeric values. We can also get a frequency distribution like Table 2.1 for this type of data but it requires more work since our data is numeric i.e. instead of just having words like male or female we may have all numbers between 18 and 50. How do we group this data?

 

Use the following steps :

 

1.    Calculate the range of the data i.e. the largest value - smallest value.

 

2.    Divide the range into a number of intervals called class intervals - the rule of thumb about the number of intervals is that it should be between 5 and 20. The width of each interval is the range divided by the number of class intervals.

 

3.    Count the number of cases falling into each interval.



 

The number of class intervals and the width of each class interval are essentially arbitrary.

 

The only rules in grouping data are that the classes must be mutually exclusive (each piece of data is placed into one and only class)

and all-inclusive (all data must be included).

 

Summarising data with Frequency Tables

 

 The following table provides the time, in days, required to complete year-end audits for a sample of 20 clients of a small public accounting firm. 

 

 

















12
 

14
 

19
 

18
 

15
 

15
 

18
 

17
 

20
 

 27
 



22
 

23
 

22
 

21
 

33
 

28
 

14
 

18
 

16
 

 13
 

 

This is called raw data

 

Remark

The highest value (i.e. the maximum)  in the data set is 33

 

The lowest value (i.e. the minimum) in the data set is 12

 

The range of our data is 33 - 12 = 21

 

 


 


We can summarise this data set by generating a frequency distribution table. We must first decide on the number of classes the frequency table will contain.

 

A general rule of thumb is to have no less than 5 and no more than 20 classes.

(Remark : roughly the square root of the number of data items)

 

With 20 values we decide to have 5 classes. (5 is close to the square root of 20).

 

 

Width

 

We must also decide the width of these classes

 

Class Width=Maximum - MinimumNumber of Classes

 

For our example, the class width  is 33 -125=215= 4.2

 

For Convenience we will use a class width of 5. If we take the first class as 10-14, note that 10 is the lower limit and 14 is the upper limit.


 [ OVERHEADS ]

 

This table provides useful information by making lists of ausit times easier to interpret.

 

However we lose some of the detail of the original list. For example, we know that there are two values between 25 and 29, but we can tell where those values lie.

 

 

  

 

Relative Frequency 

 

The relative frequency is calculated as follows

 

 

Relative Frequency=Class FrequencySum of All Frequencies

 

A relative frequency distribution shows the fraction or proportion of data items in each class; i.e. for the 10-14 class, the relative frequency is 4/20, which is 20%.

 

This can be interpreted as 20% of all data values fall between 10 and 14 inclusive.

 

[OVERHEAD]

 

A cumulative frequency distribution shows the number of data items with a value less than or equal to the upper limit for each class.

  


A cumulative relative frequency distribution shows the fraction or proportion of data items with a value less than or equal to the upper limit for each class.

 


[OVERHEAD] 

 

 



 

