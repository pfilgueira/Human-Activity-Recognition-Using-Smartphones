% Newton Raphson Method

\documentclass{beamer}

\usepackage{amsmath}
\usepackage{amssymb}
\usepackage{graphics}

\begin{document}
\begin{frame}


{
\huge
\[ \mbox{Calculus} \]
\[ \mbox{Newton-Raphson Method} \]
}
{
\Large
\[ \mbox{www.MathsResource.com} \]
}
\end{frame}%------------------------------------------------%



%------------------------------------%
%------------------------------------%
\begin{itemize}

\item[(i)]

\item[(ii)]

\item[(iii)]

\item[(iv)]

\end{itemize}

%------------------------------------%
%------------------------------------%
\begin{framed}
\begin{verbatim}


\end{verbatim}
\end{framed}
%------------------------------------%
%------------------------------------%
\newpage
%------------------------------------%
%------------------------------------%
\begin{itemize}

\item[(i)]

\item[(ii)]

\item[(iii)]

\item[(iv)]

\end{itemize}

%------------------------------------%
%------------------------------------%
\begin{framed}
\begin{verbatim}

\end{verbatim}
\end{framed}
%------------------------------------%
%------------------------------------%
\end{document}


\section{Spearman Rank Correlation}



\[ 1 - \frac{6\left( \sum d^2 + \frac{t^3-t}{12} \right)}{n(n^2-1)} \]



The adjustment for tied values
$ \frac{t^3-t}{12} $, where $t$ is the number of tied values

