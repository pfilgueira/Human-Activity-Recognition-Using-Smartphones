\documentclass{beamer}
\usepackage{amsmath}
\usepackage{amssymb}

\begin{document}
\begin{frame}
\Huge
\[\mbox{Financial Mathematics}\]
\LARGE
\[\mbox{Compound Interest}\]

\Large
\[\mbox{MathsResource.com}\]


\end{frame}


%--------------------------------------------------------------------%
\begin{frame}{MathsResource.com}{Compound Interest}
\Large
\begin{itemize}
\item Calculating Interest: Principal, Rate and Time are Known
\item 
When you know the principal amount, the rate and the time. 

\item The amount of interest can be calculated by using the formula:
$I = Prt$
\item 
For the above calculation, we have \$4500.00 to invest (or to borrow) with a rate of 9.5\% for a 6 year period of time.
\end{itemize}
\end{frame}
%--------------------------------------------------------------------%
\begin{frame}{MathsResource.com}{Compound Interest}
\Large

\textbf{Formula:}
\[ A= P(1 + r)^t \]

\begin{itemize}
\item[P] is the principal (the initial amount you borrow or deposit)
\item[r] is the annual rate of interest (percentage)
\item[t] is the number of years the amount is deposited or borrowed for.
\item[A] is the amount of money accumulated after n years, including interest.
\end{itemize}

%When the interest is compounded once a year:


%\[ A = P(1 + r)^n\]
\end{frame}
%--------------------------------------------------------------------%
\begin{frame}{MathsResource.com}{Compound Interest}
\Large
\vspace{-1cm}
However, if you borrow for 5 years the formula will look like:

\[ A = P(1 + r)^5 \]

This formula applies to both money invested and money borrowed.

\end{frame}
%-------------------------------------------------------------------%
%--------------------------------------------------------------------%
\begin{frame}{MathsResource.com}{Compound Interest}
\Large
\vspace{-1cm}
\textbf{Frequent Compounding of Interest}

What if interest is paid more frequently?
Here are a few examples of the formula:

\begin{itemize}
\item Annually = $P × (1 + r)^t$ = (annual compounding)
\item Quarterly = $P (1 + r/4)^{t/4}$ = (quarterly compounding)
\item Monthly = $P(1 + r/12)^{t/12}$ = (monthly compounding)
\end{itemize}
\end{frame}
%-------------------------------------------------------------------%

\end{document}