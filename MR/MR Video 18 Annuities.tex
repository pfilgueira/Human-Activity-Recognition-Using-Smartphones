\documentclass{beamer}

\usepackage{amsmath}
\usepackage{amssymb}

\begin{document}


\begin{frame}
\Large
\[\mbox{Annuities}\]
\end{frame}

%------------------------------------------------%
\begin{frame}
\frametitle{Annuities}
Formula for Finding the Periodic payment(R), Given A:
\[R = A/(1+(1-(1+(j/m) )^(-(n-1))/(j/m)\]
\end{frame}

%------------------------------------------------%
\begin{frame}
\frametitle{Annuities}
Examples:
Find the periodic payment of an annuity due of $\$70000$, payable annually for 3 years at 15\% compounded annually.
\[R = 70000/(1+[(1-(1+((.15)/1) )]^{(-(3-1))/((.15)/1))}\]
\[R = 70000/2.625708885\]
\[R = 26659.46724\]
\end{frame}

%------------------------------------------------%
\begin{frame}
\frametitle{Annuities}
Find the periodic payment of an annuity due of $250700, payable quarterly for 8 years at 5% compounded quarterly.
R= 250700/(1+〖(1-(1+((.05)/4) )〗^(-(32-1))/((.05)/4))
R = 250700/26.5692901
R = $9435.71
\end{frame}

%------------------------------------------------%
\begin{frame}
\frametitle{Break Even Analysis}

The Break-Even Point can alternatively be computed as the point where Contribution equals Fixed Costs.
The quantity, $\left(\text{P} - \text{V}\right)$, is of interest in its own right, and is called the Unit Contribution Margin (C): it is the marginal profit per unit, or alternatively the portion of each sale that contributes to Fixed Costs. 
\end{frame}

%------------------------------------------------%
\begin{frame}
\frametitle{Break Even Analysis}

\end{frame}



\end{document}

Maths Resource

http://www.csusm.edu/mathlab/documents/M132BusCalcFormulas%20r1-12e.pdf
http://www.math.ubc.ca/~chau/elasticity.pdf
http://www.textbooksonline.tn.nic.in/books/12/std12-bm-em-1.pdf
