\documentclass{beamer}

\usepackage{amsmath}
\usepackage{amssymb}

\begin{document}


\begin{frame}
\Large
\[
\mbox{Continuous Compounding}
\]
\end{frame}

%------------------------------------------------------------------------%
\begin{frame}
\frametitle{Weighted Average Costs of Capital}
\Large
Company XYZ  plc is a UK listed clothing retailer. 

\begin{itemize}
\item The market value of its capital
structure components is £6 billion for equity and £4 billion for debt, and its beta
coefficient is 0.85. \item The UK 3 month Treasury bill rate is 3.5\% and you estimate that
the market risk premium over and above this rate is 4.0\%. 
\item The UK corporation tax rate is 26\% ,
\item The rate paid by the company on its 10 year bonds is 5.5\%.
\end{itemize}

\end{frame}

%--------------------------------------------------------------------------%
\begin{frame}
\frametitle{Weighted Average Costs of Capital}
\Large
Calculate  XYZ plc’s weighted average cost of capital (WACC).

We start by computing GHI plc’s cost of equity capital:
\[ERi = Rf + β (ERm – Rf)\]
\[ERi = 3.5\% + 0.85 (4.0\%) = 6.9\%\]
\end{frame}

%--------------------------------------------------------------------------%
\begin{frame}
\frametitle{Weighted Average Costs of Capital}
\Large
\begin{itemize}
\item We then compute the proportions of debt and equity in the company’s capital
structure:
D / (D + E) = 40\% and E / (D + E) = 60\%
\item We can then compute its weighted average cost of capital:
\[WACC = (D / D+E) kd (1 – Tc) + (E / D + E) ke\]
\[WACC = (0.40 x 5.5\% x (1 – 0.26)) + (0.60 x 6.9\%) = 1.628\% + 4.14\% = 5.768\%\]
\end{itemize}

\end{frame}


\end{document}

Maths Resource

http://www.csusm.edu/mathlab/documents/M132BusCalcFormulas%20r1-12e.pdf
http://www.math.ubc.ca/~chau/elasticity.pdf
http://www.textbooksonline.tn.nic.in/books/12/std12-bm-em-1.pdf
