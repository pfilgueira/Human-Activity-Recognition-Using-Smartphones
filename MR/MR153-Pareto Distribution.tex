

\documentclass{beamer}

\usepackage{amsmath}
\usepackage{amssymb}
\usepackage{graphics}

\begin{document}
%-------------------------------------------------- %
\begin{frame}
\bigskip
{
\Huge
\[ \mbox{Financial Mathematics}  \]
\huge
\[ \mbox{The Pareto Distribution}  \]
}

{
\LARGE
\[ \mbox{www.MathsResource.com}  \]
%\[ \mbox{Twitter: @StatsLabDublin} \]
}
\end{frame}
%-------------------------------------------------------- %
\begin{frame}
\frametitle{The Pareto Distribution}
\Large
\vspace{-1.6cm}
Suppose the distribution of monthly salaries of full-time workers in the UK has
a Pareto distribution with minimum monthly salary $x_m = 1000$ and concentration
factor $\alpha = 3$. 

\end{frame}
%-------------------------------------------------------- %
\begin{frame}
	\frametitle{The Pareto Distribution}
	\Large
\begin{enumerate}
\item Calculate the mean monthly salary of UK full-time workers.
\item Calculate the probability that a UK full-time worker earns more than 2000 per month.
\item Calculate the median monthly salary of UK full-time workers.
\end{enumerate}
\end{frame}
%-------------------------------------------------------- %

\begin{frame}
\frametitle{The Pareto Distribution}
\Large
\vspace{-1.5cm}
The expected value of a random variable following a Pareto distribution is
\[E(X)= \begin{cases} \infty & \text{if }\alpha\le 1, \\ \frac{\alpha x_\mathrm{m}}{\alpha-1} & \text{if }\alpha>1. \end{cases}
\]

\end{frame}
%-------------------------------------------------------- %

\begin{frame}
	\frametitle{The Pareto Distribution}
	\Large
	\vspace{-2.5cm}
Because \textbf{$\alpha$} = $3$, we will use this
{
\LARGE
\[
E(X)= \frac{\alpha x_\mathrm{m}}{\alpha-1}   
\]
}
Recall that $X_m$ = 1000.
\end{frame}

%-------------------------------------------------------- %
\begin{frame}
\frametitle{The Pareto Distribution}
\Large
\vspace{-1.2cm}
The cumulative distribution function of a Pareto random variable with parameters $\alpha$ and $x_m$ is
\[
F_X(x) = \begin{cases}
1-\left(\frac{x_\mathrm{m}}{x}\right)^\alpha & \text{for } x \ge x_\mathrm{m}, \\
0 & \text{for }x < x_\mathrm{m}.
\end{cases}
\]
Using values for this example:
\[
F_X(x) = \begin{cases}
1-\left(\frac{1000}{x}\right)^3 & \text{for } x \ge 1000, \\
0 & \text{for }x < 1000.
\end{cases}
\]

\end{frame}

%-------------------------------------------------------- %
\begin{frame}
	\frametitle{The Pareto Distribution}
	\LARGE
	\vspace{-1.8cm}
	Calculate the probability that a UK full-time worker earns more than 2000 per month.
	{
		\Large
		\[
		F_X(x) = \begin{cases}
		1-\left(\frac{1000}{x}\right)^3 & \text{for } x \ge 1000, \\
		0 & \text{for }x < 1000.
		\end{cases}
		\]
	}
\end{frame}
%-------------------------------------------------------- %
\begin{frame}
	\frametitle{The Pareto Distribution}
	\LARGE
	\vspace{-1.8cm}
Calculate the probability that a UK full-time worker earns \alert{\textbf{more than}} 2000 per month.
{
	\Large
\[
F_X(x) = \begin{cases}
1-\left(\frac{1000}{x}\right)^3 & \text{for } x \ge 1000, \\
0 & \text{for }x < 1000.
\end{cases}
\]
}
\end{frame}
%-------------------------------------------------------- %
\begin{frame}
	\frametitle{The Pareto Distribution}
	\LARGE
	\vspace{-1.2cm}
Calculate the median monthly salary of UK full-time workers.

\[ \mbox{Median}: F_X(x) = 0.50\]

{
	\LARGE
	\[
	F_X(x) = \begin{cases}
	1-\left(\frac{1000}{x}\right)^3 & \text{for } x \ge 1000, \\
	0 & \text{for }x < 1000.
	\end{cases}
	\]
}

\end{frame}
%-------------------------------------------------------- %
\begin{frame}
	\frametitle{The Pareto Distribution}
	\LARGE
	\vspace{-3.2cm}
	
	\[ F_X(x) = 0.5 \qquad \rightarrow \qquad 1-\left(\frac{1000}{x}\right)^3 = 0.50\]

\vspace{2.2cm}

$ \sqrt[3]{0.5} = 0.7937 $


\end{frame}
%-------------------------------------------------------- %
\begin{frame}
	\frametitle{The Pareto Distribution}
	\LARGE
	\vspace{-3.2cm}
\[\frac{1000}{0.7937} = 1259.92\]
\end{frame}
%-------------------------------------------------------- %
\begin{frame}
	\frametitle{The Pareto Distribution}
	\LARGE
	
\end{frame}
%-------------------------------------------------------- %
\begin{frame}
	\frametitle{The Pareto Distribution}
	\LARGE
	
\end{frame}
%-------------------------------------------------------- %


\end{document}
