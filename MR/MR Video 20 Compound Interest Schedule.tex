

\documentclass{beamer}

\usepackage{amsmath}
\usepackage{amssymb}
\usepackage{graphics}

\begin{document}
%-------------------------------------------------- %
\begin{frame}
\bigskip
{
\Huge
\[ \mbox{Octave}  \]
}
{
\huge
\[ \mbox{Matrix Summations (Exercise B.2)}  \]
}

{
\LARGE
\[ \mbox{www.stats-lab.com}  \]
\[ \mbox{Twitter: @StatsLabDublin} \]
}
\end{frame}

%------------------------------------------------- %
\begin{frame}
\frametitle{Octave}
{
\Large
\begin{tabular}{|c|c|c|c|}
Year	&	Loan at Start	&	Interest	&	Loan at End	\\	\hline
0 (Now)	&	\$1,000.00	&	(\$1,000.00 × 10\% = ) \$100.00	&	\$1,100.00	\\	\hline
1	&	\$1,100.00	&	(\$1,100.00 × 10\% = ) \$110.00	&	\$1,210.00	\\	\hline
2	&	\$1,210.00	&	(\$1,210.00 × 10\% = ) \$121.00	&	\$1,331.00	\\	\hline
3	&	\$1,331.00	&	(\$1,331.00 × 10\% = ) \$133.10	&	\$1,464.10	\\	\hline
4	&	\$1,464.10	&	(\$1,464.10 × 10\% = ) \$146.41	&	\$1,610.51	\\	\hline
5	&	\$1,610.51	&		&		\\	\hline
\end{tabular} 
}
\end{frame}


An amount of $2,340.00 is deposited in a bank paying an annual interest rate of 3.1%, compounded continuously. Find the balance after 3 years. 

Solution

Use the continuous compound interest formula, A = Pe rt, with P = 2340, r = 3.1/100 = 0.031, t = 3. Recall that e stands for the Napier's number (base of the natural logarithm) which is approximately 2.7183. However, one does not have to plug this value in the formula, as the calculator has a built-in key for e. Therefore,




So, the balance after 3 years is approximately $2,568.06.

%--------------------------------------------------------------------------------------------%

http://www-stat.wharton.upenn.edu/~waterman/Teaching/IntroMath99/Class04/Notes/node12.htm


Example  2


How long does it take a principle amount to double if the interest rate is ? 

Principle is P0. 

Principle has doubled when amount at time t equals twice the principle. 

That is . 

So the question is, what value of t (how long) makes Pt double P0? 

The formula for continuous compounding states that , so the question is now ``what value of t makes ''? 

Some simplifications (trying to get at t): 


So the doubling time is ln(2)/r and this does not depend on the principle P0. 

In particular for r equal to 4% or 0.04 we get ln(2)/0.04 = 17.329. 

Always check the answers: Say P0 is 1000. 

Then 

So, it has indeed doubled (up to rounding error). 


 

\end{document}
