The Cauchy–Riemann equations on a pair of real-valued functions of two real variables $u(x,y)$ and $v(x,y)$ are the two equations:
\[\dfrac{ \partial u }{ \partial x } = \dfrac{ \partial v }{ \partial y }\]

\[\dfrac{ \partial u }{ \partial y } = -\dfrac{ \partial v }{ \partial x }\]

Typically u and v are taken to be the real and imaginary parts respectively of a complex-valued function of a single complex variable z = x + iy, f(x + iy) = u(x,y) + iv(x,y). Suppose that u and v are real-differentiable at a point in an open subset of C ( C is the set of complex numbers), which can be considered as functions from R2 to R. This implies that the partial derivatives of u and v exist (although they need not be continuous) and we can approximate small variations of f linearly. Then f = u + iv is complex-differentiable at that point if and only if the partial derivatives of u and v satisfy the Cauchy–Riemann equations (1a) and (1b) at that point. The sole existence of partial derivatives satisfying the Cauchy–Riemann equations is not enough to ensure complex differentiability at that point. It is necessary that u and v be real differentiable, which is a stronger condition than the existence of the partial derivatives, 
but it is not necessary that these partial derivatives be continuous.
