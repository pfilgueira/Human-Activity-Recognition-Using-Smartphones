\documentclass{beamer}

\usepackage{amsmath}
\usepackage{amssymb}

\begin{document}


\begin{frame}
\Large
\[
\mbox{The Sortino Ratio}
\]
\end{frame}

\begin{frame}

\frametitle{The Sortino Ratio}

The Sortino Ratio measures the risk-adjusted return of an investment asset, 
portfolio or strategy. It is a modification of the Sharpe ratio but penalizes only those returns falling below a user-specified target, or required rate of return, while the Sharpe ratio penalizes both upside and downside volatility equally. 
\end{frame}

\begin{frame}

\frametitle{The Sortino Ratio}
Though both ratios measure an investment's risk-adjusted returns, they do so in significantly different ways that will frequently lead to differing conclusions as to the true nature of the investment's return-generating efficiency. The ratio is calculated as:
\[S = \frac{R-T}{DR},\]
where 
\begin{itemize}
\item R is the asset or portfolio average realized return, 
\item T is the target or required rate of return for the investment strategy 
under consideration, (T was originally known as the minimum acceptable return, or MAR); 
\item DR is the target semi-deviation given by
\end{itemize}
\end{frame}
%---------------------------------------------------

\begin{frame}

\frametitle{The Sortino Ratio}
The target semi-deviation is given by
\[ DR = \left( \, \frac{1}{N} \, \sum_{k=1}^N (r_k - T)^2\,f(r_k)\, \right)^{1/2},\] 
where
T is a target rate of return $r_k$ is the $k^{th}$ returnf(r_k) = 1  if $r_k < T$ and 
$f(r_k) = 0 $ if $r_k > T$

This can be thought of as the root mean squared underperformance, where the underperformance is the amount by which a return is below target (and returns above target are treated as underperformance of 0).

\end{frame}



\end{document}

Maths Resource

http://www.csusm.edu/mathlab/documents/M132BusCalcFormulas%20r1-12e.pdf
http://www.math.ubc.ca/~chau/elasticity.pdf
http://www.textbooksonline.tn.nic.in/books/12/std12-bm-em-1.pdf
