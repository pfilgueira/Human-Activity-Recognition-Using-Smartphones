

\documentclass{beamer}

\usepackage{amsmath}
\usepackage{amssymb}
\usepackage{graphics}

\begin{document}
%-------------------------------------------------- %
\begin{frame}
\bigskip
{
\Huge
\[ \mbox{Financial Mathematics}  \]
\huge
\[ \mbox{The Pareto Distribution}  \]
}

{
\LARGE
\[ \mbox{www.Stats-Lab.com}  \]
\[ \mbox{Twitter: @StatsLabDublin} \]
}
\end{frame}
%-------------------------------------------------------- %
\begin{frame}
\frametitle{The Pareto Distribution}
\Large

Suppose the distribution of monthly salaries of full-time workers in the UK has
a Pareto distribution with minimum monthly salary $x_m = 1000$ and concentration
factor $\alpha = 3$. 

\begin{enumerate}
\item Calculate the mean monthly salary of UK full-time workers.
\item Calculate the probability that a UK full-time worker earns more than 2000 per month.
\item Calculate the median monthly salary of UK full-time workers.
\end{enumerate}
\end{frame}
%-------------------------------------------------------- %

\begin{frame}
\frametitle{The Pareto Distribution}
\Large
\vspace{-2cm}
The expected value of a random variable following a Pareto distribution is
\[E(X)= \begin{cases} \infty & \text{if }\alpha\le 1, \\ \frac{\alpha x_\mathrm{m}}{\alpha-1} & \text{if }\alpha>1. \end{cases}
\]
\end{frame}
%-------------------------------------------------------- %
\begin{frame}
\frametitle{The Pareto Distribution}
\Large
\vspace{-1.2cm}
The cumulative distribution function of a Pareto random variable with parameters $\alpha$ and $x_m$ is
\[
F_X(x) = \begin{cases}
1-\left(\frac{x_\mathrm{m}}{x}\right)^\alpha & \text{for } x \ge x_\mathrm{m}, \\
0 & \text{for }x < x_\mathrm{m}.
\end{cases}
\]
Using values for this example:
\[
F_X(x) = \begin{cases}
1-\left(\frac{1000}{x}\right)^3 & \text{for } x \ge 1000, \\
0 & \text{for }x < 1000.
\end{cases}
\]
\end{frame}
%-------------------------------------------------------- %
\end{document}