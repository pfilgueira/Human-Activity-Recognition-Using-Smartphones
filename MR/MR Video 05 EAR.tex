
\documentclass{beamer}

\usepackage{amsmath}
\usepackage{amssymb}
\usepackage{graphics}

\begin{document}
%-------------------------------------------------- %
\begin{frame}
\bigskip
{
\Huge
\[ \mbox{Financial Mathematics}  \]
\huge
\[ \mbox{Effective Annual Interest Rate}  \]
}

{
\LARGE
\[ \mbox{www.Stats-Lab.com}  \]
\[ \mbox{Twitter: @StatsLabDublin} \]
}
\end{frame}

%------------------------------------------------- %
\begin{frame}
\frametitle{Effective Annual Interest Rate}
\Large
An investment's effective annual interest rate is the percentage gain in value accrued over a year when compounding takes place more often than once a year. 

Calculated as the following:

\[ \mbox{EAR} = \left( 1 + \frac{r}{n}  \right)^n -1 \]

r: stated annual interest rate
n: number of compounding periods in a year.
\end{frame}


%===================================================================================================================%


What it is: 


The effective annual interest rate is the rate of interest an investor earns in a year after accounting for the effects of compounding. 
 

%--------------------------------------%



% Formula 


The formula for effective annual interest rate is:
 
\[ (1 + i / n)^n - 1 \]
 
Where: 
 
i = the stated annual interest rate
 
n = the number of compounding periods in one year

%--------------------------------------%
 
For example, let’s assume you buy a certificate of deposit with a 12% stated annual interest rate. If the bank compounds the interest every month (that is, 12 times per year), then using this information and the formula above, the effective annual interest rate on the CD is:
 
(1 + .12/12)12 - 1 = .12683 or 12.683%
 
Let’s look at it from another angle. Let’s assume you put $1,000 into the 12% CD. Over 12 months, the investment will look like this:



The percentage change from $1,000 to $1,126.83 is ($1,126.83 - $1,000)/$1,000 = .12683 or 12.683%. Even though the bank has advertised a 12% interest rate, your money actually grew by 12.683%.
 


%--------------------------------------%


The effective annual interest rate takes compounding into consideration and is thus almost always higher than the stated annual interest rate. 
It is a useful tool for evaluating the true return on an investment or the true interest rate paid on a loan.


%===================================================================================================================%

\end{document}
