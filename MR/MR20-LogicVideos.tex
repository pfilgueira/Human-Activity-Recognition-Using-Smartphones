%---------------------------------------------------------%
3.5 Exercises 3
l. The following propositions relate to a 3-bit binary string s.
\begin{itemize}
\item[(p)] Only one bit ofs is 0.
\item[(q)] The first two bits ofs are the same.
\end{itemize}

Find the truth set for each of the following statements:
r; q; PM; PV¢1·
%---------------------------------------------------------%
2. Let p, q denote the following propositions concerning an integer n.

p:n$50; q: 222}.0.

Express in words, as simply as you can, the following statements as conditions ou 11.
%rip; r/M1: *(··<1); ··2¤V—q·
%---------------------------------------------------------%
3, Let p, q be the following propositions.

\begin{itemzie}
\item[p]: This book is on Databases.
\item[q]: This bool: is on Programming.
\end{itemize}
%-----------%
Express each of the following cornpoundstatements symbolically in TWO different ways:
\begin{itemize}
\item[(a)] This book is not on Databases or Programming.
\item[(b)] This book is not on Databases and Programming.
\end{itemize}
%---------------------------------------------------------%
4. Use truth tables to prove that (p A q) V (-·pA q) : q. (Hint: you will need to construct
columns for p, q and p A q,-·p, —·pA q, (p A q) V (—·pA q). Remember to make a comment at
the end to say why the table proves that the two statements are logically equivalent.)

%----------------------------------------------------------%
5. Construct a truth table for each ofthe following compound statements and hence find simpler
propositions to which each is equivalent.
(a)p\/F; (b)p/\T.
%---------------------------------------------------------%
6. Let pi 9, r be the following propositions concerning an integer n.
pz n : 20; q: rz is even; ra n is positive.
Express each of the following conditional statements symbolically, using the symbol —>.
(aj If rr : 20, then n is positive.
(b) n is even if n : 20.
(c) n : 20 only if n is even.
%---------------------------------------------------------%
7. Let q and r be the propositions defined in the previous exercise. Complete the following table
by giving the truth value of each ofthe statements q, r. $q \rightarrow r$, $r \rigtharrow q$ and q 4-+ r corresponding
to each value of n.
n    &   q  & r  $q \rightarrow r$   &    r-—·>q   &q<—>T
-8
-3
10
lT
%---------------------------------------------------------%
8. Use truth tables to prove that -·p 4-+ ··»q is logically equivalent to p e·> q.
Q. Write the contrapositive of each of the following statements.
\begin{itemize}
\item[(a)] li` n : 12, then n is divisible by 3.
\item[(b)] lf rr : 5, then rz is positive.
\item[(c)] lf the quadrilateral is a square, then its four sides are equal.
\end{itemize}
%---------------------------------------------------------%
l0. The basis for logical argument is that given propositions p,q, r such that p imples q and q
implies r, then we can deduce that p implies r. The validity of this argument depends upon
the fact that the statement [(11 ··> q) \wedge (q —> r)] -+ (p —> r) is always true, that is, it is a
tautology.
Construct a truth table with columns for p,q, r and p -> q,q ·—> r, (p —> q) /\ (q —> r),p —>
r, Kp —-> q) /\ (q -->   —> (p -—> r). Hence prove that
l(r>—+·1)/\(<1—>r)] —> (r¤—>r)
is indeed a tautology.
%---------------------------------------------------------%
ll. The following logic network accepts inputs p and q, which may each independently have the
value O or l.
P   ?
I1. »
‘? ..4T
Q D
(aj Copy the network and label each of the gates appropriately with one of the words
“NOT”, “AND” or "OR”. Label the diagram also with a symbolic expression for the
output from each gate.
(b) Construct a logic table to show the value of the output corresponding to each combina»
tion of values (O or 1) for the inputs p and q.
[c] Find a simpler expression that is logically equivalent to the final output,

