
MA4413 Exam Questions


Huffman Coding

R Code for Binary Channel


\end{frame}
%-------------------------------------------------- %
\begin{frame}

The input source to a noisy communication channel is a random variable X over the

four symbols a, b, c, d. The output from this channel is a random variable Y over these same

four symbols. The joint distribution of these two random variables is as follows:

\end{frame}
%-------------------------------------------------- %
\begin{frame}

(a) Write down the marginal distribution for X and compute the marginal entropy H(X) in bits.

(b) Write down the marginal distribution for Y and compute the marginal entropy H(Y ) in bits.

(c) What is the joint entropy H(X, Y ) of the two random variables in bits?

(d) What is the conditional entropy H(Y |X) in bits?
(e) What is the mutual information I(X; Y ) between the two random variables in bits?


\end{frame}
%-------------------------------------------------- %
\begin{frame}
Consider a data source from the alphabet (A,B,C,D) with probability distribution (0.3,0.4,0.2,0.10)

•
Derive a fixed length binary code and a huffman code for the source.

•
Comment in the difference in the shape of a binary tree representing the fixed length code and the Huffman tree of the source.

•
Discuss which code is closer to optimal. Show all your work to justify your answer.

\end{frame}
%-------------------------------------------------- %
\begin{frame}
John claims that he can find a uniquely decodable binary code for any four symbols (A,B,C,D) with codeword lengths (2,1,3,3) respectively.

Explain why we should or should not believe John.


Give an exmple of such a uniquely decodable binary code to support your argument.


\end{frame}
%-------------------------------------------------- %
\begin{frame}

Huffman Coding


http://books.google.ie/books?id=9KtFcFa81FcC&pg=PA250&lpg=PA250&dq=huffman+code+%2B+schaum&source=bl&ots=ym6dDRlJlc&sig=XVJ-iVbQqxYa3xFZDnKq9RTiRX4&hl=en&ei=88-1TsWTDJGKhQfvq9yGBA&sa=X&oi=book_result&ct=result&resnum=3&ved=0CCcQ6AEwAg#v=onepage&q&f=false


A: 22

B: 5

C: 11

D: 19

E: 2

F: 11

G: 25

H: 5


A: 00

B: 11011

C: 011

D: 111

E: 111

F: 010

G: 10

H: 1100



--------------------------------------------------------------------------------


R Code for Binary Channel


X1 =c(100,70)

Y1 =c(200,70)

X2 =c(100,20)

Y2 =c(200,20)


XS =c(100,100,200,200)

YS =c(70,20,70,20)

plot(XS,YS,pch=16,xlim=c(75,225),ylim=c(0,100))


#line 1





lines(c(100, 200),c( 70, 70))

lines(c(100, 200),c( 20, 20))

lines(c(100, 200),c( 20, 70))

lines(c(100, 200),c( 70, 20))


arrows(100, 70, 180, 70, length = 0.15, angle = 25, code = 2,col = par("fg"), lty = par("lty"), lwd = par("lwd"))

arrows(100, 20, 180, 20, length = 0.15, angle = 25, code = 2,col = par("fg"), lty = par("lty"), lwd = par("lwd"))

arrows(100, 20, 180, 60, length = 0.15, angle = 25, code = 2,col = par("fg"), lty = par("lty"), lwd = par("lwd"))

arrows(100, 70, 180, 30, length = 0.15, angle = 25, code = 2,col = par("fg"), lty = par("lty"), lwd = par("lwd"))


text(100,73,"X1", font=4)

text(200,73,"Y1", font=4)

text(100,17,"X2", font=4)

text(200,17,"Y2", font=4)




text(172,73,"0.9", font=4)

text(172,17,"0.8", font=4)

text(172,30,"0.1", font=4)

text(172,60,"0.2", font=4)


