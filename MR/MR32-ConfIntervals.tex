


--------------------------------------------------------------------------------
Type I and II Error

Explain what is meant by the terms Type I error and Type II error.


%==============================================================================%
\begin{frame}
\frametitle{Confidence Intervals for Proportions}
\Large
\begin{itemize}
\item In a survey conducted by a mail order company a random sample of 200 customers yielded 172 who indicated that they 
were highly satisfied with the delivery time of their orders. 

\item Calculate an approximate 95\% confidence interval for the proportion of the company's customers who are 
highly satisfied with delivery times.
\end{itemize}
\end{frame}
%==============================================================================%
\begin{frame}
\frametitle{Confidence Intervals for Proportions}

\[p=172200= 86\%\]


\[ \frac{p(100-p)}{n} =\frac{86 \times 14}{200}\]


\end{frame}
%==============================================================================%
\begin{frame}
\frametitle{Confidence Intervals for Proportions}




HT for Difference of Props


A survey, carried out at a major flower and gardening show, was concerned with the association between the intention to return to the show next year and the purchase of

goods at this year s show. There were 220 people interviewed and of these 101 had made a purchase; 69 of these people said they intended to return next year. Of the

119 who had not made a purchase, 68 said they intended to return next year.


By testing the difference between the proportions of purchasers and non-purchasers who intend to return next year, examine whether there is sufficient

evidence to justify concluding that the intention to return depends on whether or not a purchase was made.



H0: population proportions of those who intend to return are equal

v H1: not H0

Proportion of purchasers 1 69 /101; proportion of non-purchasers

2 68 /119



Observed value of D = 0.1117


Estimated standard error of D = 6.558%

