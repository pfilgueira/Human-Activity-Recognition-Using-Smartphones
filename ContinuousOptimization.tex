Continuous optimization is a branch of applied mathematics in the field of optimization, which refers to selecting the greatest element from a large set of alternative options. This kind of optimization is different from discrete optimization in that variables used in an objective function are able to assume values that are real, such as interval values from a real line. Continuous optimization is applied to many different fields and disciplines, including computer science, market analysis, and microeconomics. It is also an important aspect in the broader field of mathematics.

In computer science, continuous optimization is used for many different things, including instruction streams in an application. Programmers use a dynamic optimizer which is hardware-based to optimize a certain application in a continuous fashion. The hardware is simple and table-based, being used and placed in certain stages for dataflow optimization functions. A continuous optimizer creates a reduction in dataflow height, performing constant and consistent propagation, elimination of redundant loads, reassociation, removal of silent stores and forwarding of stores. An optimization performance’s impact is enhanced by integrated values that are generated from units that are executed back to the same process of optimization.

What this allows is the execution of continuous optimization time, which is made of input values of the instructions within the optimizer. This leaves a smaller amount of work for portions of the program’s pipeline that are not in order. Continuous optimization is also able to detect false predictions of branches much earlier, which creates a reduction in the penalty of false predictions. This is quite useful in the realm of computer science and is used in entities such as mediabranch workloads, SPECint, and SPECfp. The optimizer function has been found to execute at a 33% success rate and resolve issues at a 29% success rate.

Another field of study that uses continuous optimization is marketing analysis and microeconomics, especially as it relates to small, isolated customer demographics and markets. Successful analysts use continuous optimization to determine their values regarding customers by following them online as well as off. There are certain open source software programs that allow these analysts to plug in values or follow demographics in certain areas. What these analysts hope to achieve is to reduce the costs of maintenance and implementation by leveraging certain sets of tags as well as creating a particular unified infrastructure to serve all potential marketing campaigns. They seek to analyze the data at hand and use it to optimize the efficiency of their marketing.
