Stochastic modeling is a technique of presenting data or predicting outcomes that takes into account a certain degree of randomness, or unpredictability. The insurance industry, for example, depends greatly on stochastic modeling for predicting the future condition of company balance sheets, since these may depend on unpredictable events resulting in the paying of claims. Many other industries and fields of study can benefit from stochastic modeling, such as statistics, stock investing, biology, linguistics, and quantum physics.

Especially in the world of insurance, stochastic modeling is crucial in determining what outcomes may be expected, versus which ones are unlikely. Rather than using fixed variables such as in other mathematical modeling, a stochastic model incorporates random variations to predict future conditions and to see what they might be like. Of course, the possibility of one random variation implies that many could occur. For this reason, stochastic models are not run just once, but hundreds or even thousands of times. This larger collection of data not only expresses which outcomes are most likely, but what ranges can be expected as well.

To understand the idea of stochastic modeling, it may be helpful to consider that it is the opposite, in a way, of deterministic modeling. This second type of modeling is what most of elementary mathematics consists of. The solution to a problem can usually only have one right answer, and the graph of a function can only have one specific set of values. Stochastic modeling, on the other hand, is like varying a complicated math problem slightly to see how the solution is affected, and then doing so many times and in different ways. These slight variations represent the randomness or unpredictability of real-world events and their effects.

Another real-world application of stochastic modeling, besides insurance, is manufacturing. Manufacturing is seen as a stochastic process because of the effect that unknown or random variables can have on the end result. For example, a factory which makes a certain product will always find that a small percentage of the products do not come out as intended, and cannot be sold. This may be due to a variety of factors, such as the quality of inputs, the working condition of the production machinery, and the competence of employees, among others. The unpredictability of how these factors affect outcomes can be modeled to predict a certain error rate in manufacturing, which can be planned ahead for.
